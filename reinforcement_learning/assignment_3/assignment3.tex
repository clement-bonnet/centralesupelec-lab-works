\documentclass[a4paper]{article}
\usepackage[english]{babel}
\usepackage[utf8]{inputenc}
\usepackage{fancyhdr}
\usepackage{hyperref}
\usepackage{amsmath,amsfonts,amssymb,amsthm}
\usepackage[a4paper, bottom=1.3in, top=1.3in, right=1in, left=1in]{geometry}
\usepackage[usenames,dvipsnames]{xcolor}
\usepackage[lined,boxed]{algorithm2e}
\usepackage{natbib}
\usepackage{dsfont}
\usepackage{tikz}
\usetikzlibrary{calc}
\definecolor{amaranth}{rgb}{0.9, 0.17, 0.31}
\newcommand{\rcol}[1]{{\color{amaranth}#1}}

\usepackage{todonotes}
\newcommand{\todomp}[1]{\todo[color=Green!10, inline]{\small MP: #1}}
\newcommand{\todompout}[1]{\todo[color=Green!10]{\scriptsize MP: #1}}

\newcommand{\wh}[1]{\widehat{#1}}
\newcommand{\wt}[1]{\widetilde{#1}}
\newcommand{\transp}{\intercal}

%%%%%%%%%%%%%%%%%%%%%%%%%%%%%%%%%%%%%%%%%%%
% Insert your name here
\newcommand{\fullname}{Clément Bonnet}
\usepackage{caption}
\usepackage{subcaption}
\usepackage{float}
\DeclareUnicodeCharacter{2212}{-}
%%%%%%%%%%%%%%%%%%%%%%%%%%%%%%%%%%%%%%%%%%%

\newcommand{\lecture}[3]{
   \pagestyle{myheadings}
   \thispagestyle{plain}
   \newpage
   \setcounter{page}{1}
   \noindent
   \begin{center}
   \framebox{
      \vbox{\vspace{2mm}
              \hbox to .97\textwidth { {\bf MVA: Reinforcement Learning (2020/2021) \hfill Homework 3} }
       \vspace{6mm}
       \hbox to .97\textwidth { {\Large \hfill #1 \hfill } }
       \vspace{6mm}
       \hbox to .97\textwidth { {Lecturers: \it A. Lazaric, M. Pirotta  \hfill {{\footnotesize(\today)}}} }
      \vspace{2mm}}
   }
   \end{center}
   Solution by {\color{amaranth}\fullname}
   \markboth{#1}{#1}
   \vspace*{4mm}
}


\DeclareMathOperator*{\argmax}{\arg\,\max}
\DeclareMathOperator*{\argmin}{\arg\,\min}
\DeclareMathOperator*{\arginf}{\arg\,\inf}


\setlength{\parindent}{0cm}
\begin{document}
\lecture{Exploration in Reinforcement Learning (theory)}{3}


\pagestyle{fancy}
\fancyhf{}
\rhead{Full name: {\color{amaranth}\fullname}}
\lhead{Exploration in Reinforcement Learning}
\cfoot{\thepage}


\section{UCB}

We find ourselves in the setting of multi-arm bandits.
\begin{align*}
	S_{j,t} &= \sum_{k=1}^t X_{i_k,k} \cdot \mathds{1}(i_k = j)\\
	N_{j,t} &= \sum_{k=1}^t \mathds{1}(i_k = j)\\
	\wh\mu_{j,t} &= \frac{S_{j,t}}{N_{j,t}}\\
\end{align*}
The question is to prove whether or not $\wh\mu_{j,t}$ is an unbiased estimator of $\mu_j$.

At first sight, one could interpret $\wh\mu_{j,t}$ as the simple mean estimate of $\mu_j$ and thus would be unbiased. However, this would only apply if samples $X_{i_k,k}$ were independent and identically distributed (iid), which is not the case here in the online on-policy learning of UCB. Whether an arm is pulled or not depends on previous samples and therefore one can expect the estimate to rather have some bias.

To prove the biasedness of $\wh\mu_{j,t}$, or rather to show that it is not unbiased in the general case, we will consider a simple case and compute its analytical bias. Let us consider the setting of Bernoulli bandits as in section \ref{sec:bernoulli} with $k=2$ binary arms of parameters $\mu_1$ and $\mu_2$. One pulls the arm $i_t$ such that
\[
i_t \in \argmax_{j} \wh\mu_{j,t} + U(N_{j,t}, \delta)
\]
We assume here that arms are pulled \textbf{randomly} in case of a tie. The UCB exploration term is infinite for $t \in \{1,2\}$ where both arms are pulled successively. At $t=3$, both arms have been pulled once and one of them is going to be pulled again. We look at the sample mean estimates $\wh\mu_{1,3}$ and $\wh\mu_{2,3}$ after the third action.
\begin{align*}
	\mathbb{P}\left(\wh\mu_{1,3} = \frac{1}{2}\right) &= (1-\mu_1)(1-\mu_2)\frac{\mu_1}{2} + \mu_1(1-\mu_1)(1-\mu_2) + \mu_1\mu_2\frac{1-\mu_1}{2} \\
	&= \mu_1\left(\frac{3}{2} - \frac{3}{2}\mu_1 - \mu_2 + \mu_1\mu_2\right) \\
	\mathbb{P}\left(\wh\mu_{1,3} = 1\right) &= \mu_1^2(1-\mu2) + \mu_1\mu_2(1-\frac{1+\mu_1}{2}) \\
	&= \mu_1\left(\mu_1 + \frac{1}{2}\mu_2 - \frac{1}{2}\mu_1\mu_2\right) \\
\end{align*}
This leads to the calculation of the expected value of $\wh\mu_{1,3}$,
\begin{align*}
	\mathbb{E}_{UCB}\left[\wh\mu_{1,3}\right] &= \frac{1}{2}\mathbb{P}\left(\wh\mu_{1,3} = \frac{1}{2}\right) + \mathbb{P}\left(\wh\mu_{1,3} = 1\right) \\
	&= \mu_1\left(1 - \frac{1}{4}\left(1-\mu_1\right)\right)
\end{align*}
The bias of arm $1$ is therefore:
\[
	\boxed{\text{bias}_1 \equiv \mathbb{E}_{UCB}\left[\wh\mu_{1,3}\right] - \mu_1 = -\frac{1}{4}\mu_1(1-\mu_1)}
\]
Since the arms play a symmetrical role in the derivation of the bias, one can derive the bias for arm $2$:
\[
\boxed{\text{bias}_2 \equiv \mathbb{E}_{UCB}\left[\wh\mu_{2,3}\right] - \mu_2 = -\frac{1}{4}\mu_2(1-\mu_2)}
\]
These biases are strictly negative if $\; 0 < \mu_1,\mu_2 < 1 $. Therefore, $\boxed{\wh\mu_{j,t} \text{ is not an unbiased estimator of } \mu_j \text{ in general}}$.



\section{Best Arm Identification}

\begin{itemize}
	\item Let us compute a function $U(t,\delta)$ that satisfies the any-time confidence bound. For any arm $i \in [k]$
	\[
		\mathbb{P}\left(\bigcup_{t=1}^{\infty} \left\{ | \wh{\mu}_{i,t} - \mu_i | > U(t,\delta)\right\} \right) \leq \delta
	\]
	If one chooses $\boxed{U(t,\delta) = \sqrt{\frac{1}{2t} \log \frac{\pi^2 t^2}{3\delta} }}$,
	\begin{align*}
		\mathbb{P}\left(\bigcup_{t=1}^{\infty} \left\{ | \wh{\mu}_{i,t} - \mu_i | > U(t,\delta)\right\} \right)
		& \leq \sum_{t=1}^\infty \mathbb{P}\left( \left\{ | \wh{\mu}_{i,t} - \mu_i | > U(t,\delta)\right\} \right) \\
		& \leq \sum_{t=1}^\infty 2\exp(-2tU(t, \delta)^2) \quad \text{(Hoeffding's inequality)} \\
		& = \sum_{t=1}^\infty 2\exp\left(-\log \frac{\pi^2 t^2}{3\delta}\right) \\
		& = \sum_{t=1}^\infty \frac{6\delta}{\pi^2}\frac{1}{t^2} \\
		& = \delta
	\end{align*}
	
	\item Let $\mathcal{E} = \bigcup_{i=1}^{k}\bigcup_{t=1}^{\infty} \left\{ | \wh{\mu}_{i,t} - \mu_i | > U(t,\delta')\right\}$. For $\boxed{\delta' = \frac{\delta}{k}}$,
	\begin{align*}
		\mathbb{P}\left(\mathcal{E} \right)
		& = \sum_{i=1}^k \mathbb{P}\left(\bigcup_{t=1}^{\infty} \left\{ | \wh{\mu}_{i,t} - \mu_i | > U(t,\delta')\right\} \right) \\
		& \leq \sum_{i=1}^k \delta' \\
		& = \sum_{i=1}^k \frac{\delta}{k} \\
		& = \delta
	\end{align*}
	Therefore, $\mathbb{P}\left(\mathcal{E} \right) \leq \delta$. This is a bad event since the confidence intervals do not hold.
	
	\item Let us show that with probability at least $1-\delta$, the optimal arm $i^\star =\argmax_i \{\mu_{i}\}$ remains in the active set $S$.
	
	Let us assume $\neg \mathcal{E}$. Under such conditions,
	\[
		\forall t, \forall i, | \wh{\mu}_{i,t} - \mu_i | \leq U(t,\delta')
	\]
	Therefore, 
	\begin{equation*}
		\forall t, \forall i \neq i^\star,
		\begin{cases}
			& \wh{\mu}_{i^\star,t} \geq \mu^\star - U(t,\delta')\\
			& \wh{\mu}_{i,t} \leq \mu_i + U(t,\delta')\\
		\end{cases}
	\end{equation*}
	\begin{equation}
		\label{equ:neg_e}
		\forall t, \forall i \neq i^\star,\quad
		\wh{\mu}_{i^\star,t} - \wh{\mu}_{i,t} \geq \Delta_i - 2U(t,\delta')
	\end{equation}
	Let us now show that this implies that in such conditions, arm $i^\star$ remains in the active set $S$. Under such conditions, let us assume the opposite and prove by contradiction. Assume the arm $i^\star$ is eliminated at time $t_0$. Using $\delta'$ instead of $\delta$ in the algorithm, this means:
	\begin{equation}
		\label{equ:contradiction}
		\exists i_0 \neq i^\star, \wh{\mu}_{i^\star,t_0} \leq \wh{\mu}_{i_0,t_0} - 2U(t_0,\delta')
	\end{equation}
	Using equation (\ref{equ:neg_e}) for $t=t_0$ and $i=i_0$ combined with equation (\ref{equ:contradiction}), one finds the following inequality:
	\begin{equation*}
		 \Delta_{i_0} - 2U(t_0,\delta') \leq \wh{\mu}_{i^\star,t_0} - \wh{\mu}_{i_0,t_0} \leq -2U(t_0,\delta')
	\end{equation*}
	This implies $\Delta_{i_0} \leq 0$ which is a contradiction since $\Delta_{i_0} = \mu^\star - \mu_{i_0} > 0$ (we assume for simplicity there is only one best arm). Therefore, by contradiction, we prove that under $\neg \mathcal{E}$ conditions, the arm $i^\star$ remains in the active set $S$.
	\[
		\neg \mathcal{E} \subset \left\{ \text{arm $i^\star$ remains in the active set} \right\}
	\]
	\[
		\mathbb{P}\left(\neg \mathcal{E} \right) \leq \mathbb{P}\left( \left\{ \text{arm $i^\star$ remains in the active set} \right\} \right)
	\]
	\[
		\boxed{\mathbb{P}\left( \left\{ \text{arm $i^\star$ remains in the active set} \right\} \right) \geq 1 - \mathbb{P}\left( \mathcal{E} \right) \geq 1 - \delta}
	\]

	\item Under event $\neg \mathcal{E}$, let us find $C_1$ such that for an arm $i \neq i^\star$, if $\Delta_i \geq C_1 U(t, \delta')$, then the arm $i$ will be removed from the active set.
	
	Let $i \neq i^\star$ and apply $\neg \mathcal{E}$ conditions on $i$ and $i^\star$.
	\begin{align*}
		&\begin{cases}
			& | \wh{\mu}_{i^\star,t} - \mu^\star | \leq U(t,\delta')\\
			& | \wh{\mu}_{i,t} - \mu_i | \leq U(t,\delta')\\
		\end{cases}\\
		&\begin{cases}
			& -U(t,\delta') + \mu^\star \leq \wh{\mu}_{i^\star,t} \leq \mu^\star + U(t,\delta')\\
			& -U(t,\delta') - \mu_i \leq -\wh{\mu}_{i,t} \leq -\mu_i + U(t,\delta')\\
		\end{cases}\\
	\end{align*}
	\[
		\neg \mathcal{E} \implies \Delta_i - 2U(t,\delta') \leq \wh{\mu}_{i^\star,t} - \wh{\mu}_{i,t} \leq \Delta_i + 2U(t,\delta')
	\]
	According to the algorithm (using $\delta'$ and not $\delta$ in the pseudo-code), if $\wh{\mu}_{i^\star,t} - \wh{\mu}_{i,t} \geq 2U(t,\delta')$, the arm $i$ will be removed from the active set.
	
	Therefore, if $\Delta_i - 2U(t,\delta') \geq 2U(t,\delta') \iff \Delta_i \geq 4U(t,\delta')$, the arm $i$ will be removed for sure from the active set, under $\neg \mathcal{E}$ conditions.
	
	Under event $\neg \mathcal{E}$, an arm $i \neq i^\star$ will be removed from the active set when $\boxed{\Delta_i \geq C_1 U(t, \delta')\; \text{with}\; C_1=4}$.
	
	With our definition of $U(t, \delta')$,
	\[
		\Delta_i \geq 4 U(t, \delta') \iff \Delta_i^2 \geq \frac{8}{t} \left( 2\log t + \log \frac{\pi^2}{3\delta} \right)
	\]
	By minimizing $\log t$ by $0$ (since $t \geq 1$), for every arm $i \neq i^\star$,
	\[
		\boxed{t \geq \frac{8\log \frac{\pi^2}{3\delta}}{\Delta_i^2}} \implies \Delta_i \geq 4 U(t, \delta') \implies \text{arm $i$ will be removed}
	\]
	
	\item Let us compute a lower bound on the sample complexity for identifying the optimal arm with probability $1-\delta$.
	
	\[
		\left( \forall i \neq i^\star,\; t \geq \frac{8\log \frac{\pi^2}{3\delta}}{\Delta_i^2} \right) \iff t \geq \frac{8\log \frac{\pi^2}{3\delta}}{\Delta_{i^\star}^2}
	\]
	With $\Delta_{i^\star} = \min_{i \neq i^\star} \Delta_i$.
	\[
		\boxed{\tau_\delta \geq \frac{8\log \frac{\pi^2}{3\delta}}{\Delta_{i^\star}^2}} \quad \text{with probability $1-\delta$.}
	\]
	
\end{itemize}

\section{Bernoulli Bandits}
\label{sec:bernoulli}

UCB and KL-UCB algorithms for Bernoulli Bandits have been implemented in Python, using NumPy and Matplotlib libraries. Expected regret of both algorithms are plotted in figure \ref{fig:regret} in the case of $k=2$ Bernoulli arms of means $\mu_1 \in \{0.1,0.5,0.9\}$ and $\mu_2 = 0.5 + \Delta$ with $\Delta \in [-0.5,0.5]$.

First, one must observe that both algorithms always have a regret of $0$ when $\mu_1 = \mu_2$ (corresponding to $\Delta = 0$ in (a), $\Delta = -0.4$ in (b) and $\Delta = 0.4$ in (c)). Indeed this is explained by both arms having the same expected value and thus they are equally good in average. This means that whatever choice one makes, there is no regret from it.

Then, both algorithms perform the worst when $\mu_1 \approx \mu_2$ but $\mu_1 \neq \mu_2$. This is straightforward to understand since when their expected values are very close to each other, it is harder or at least it takes longer to distinguish them. Therefore, one makes many more errors in choosing the wrong arm, increasing the regret.

Finally, one can see in figure \ref{fig:regret} that the KL-UCB algorithm performs better than the UCB one. Although for $\mu_1 = 0.5$, KL-UCB and UCB tend to have rather similar performances when $\mu_2$ remains close to $\mu_1$, KL-UCB performs significantly better when $\mu_1 \in \{0.1,0.9\}$. The Kullback-Leibler divergence is indeed better at distinguishing between two Bernoulli distributions that would have their means close to 0 or close to 1. This is why KL-UCB performs better than UCB in figure \ref{fig:regret} (b) and (c) when $\mu_1 \approx \mu_2$ whereas it performs closely to UCB in (a) since the KL divergence is smaller when both means are around $0.5$.

\begin{figure}[H]
	\centering
	\begin{subfigure}[h]{\textwidth}
		\begin{center}
			%% Creator: Matplotlib, PGF backend
%%
%% To include the figure in your LaTeX document, write
%%   \input{<filename>.pgf}
%%
%% Make sure the required packages are loaded in your preamble
%%   \usepackage{pgf}
%%
%% and, on pdftex
%%   \usepackage[utf8]{inputenc}\DeclareUnicodeCharacter{2212}{-}
%%
%% or, on luatex and xetex
%%   \usepackage{unicode-math}
%%
%% Figures using additional raster images can only be included by \input if
%% they are in the same directory as the main LaTeX file. For loading figures
%% from other directories you can use the `import` package
%%   \usepackage{import}
%%
%% and then include the figures with
%%   \import{<path to file>}{<filename>.pgf}
%%
%% Matplotlib used the following preamble
%%
\begingroup%
\makeatletter%
\begin{pgfpicture}%
\pgfpathrectangle{\pgfpointorigin}{\pgfqpoint{6.000000in}{4.000000in}}%
\pgfusepath{use as bounding box, clip}%
\begin{pgfscope}%
\pgfsetbuttcap%
\pgfsetmiterjoin%
\pgfsetlinewidth{0.000000pt}%
\definecolor{currentstroke}{rgb}{1.000000,1.000000,1.000000}%
\pgfsetstrokecolor{currentstroke}%
\pgfsetstrokeopacity{0.000000}%
\pgfsetdash{}{0pt}%
\pgfpathmoveto{\pgfqpoint{0.000000in}{0.000000in}}%
\pgfpathlineto{\pgfqpoint{6.000000in}{0.000000in}}%
\pgfpathlineto{\pgfqpoint{6.000000in}{4.000000in}}%
\pgfpathlineto{\pgfqpoint{0.000000in}{4.000000in}}%
\pgfpathclose%
\pgfusepath{}%
\end{pgfscope}%
\begin{pgfscope}%
\pgfsetbuttcap%
\pgfsetmiterjoin%
\definecolor{currentfill}{rgb}{1.000000,1.000000,1.000000}%
\pgfsetfillcolor{currentfill}%
\pgfsetlinewidth{0.000000pt}%
\definecolor{currentstroke}{rgb}{0.000000,0.000000,0.000000}%
\pgfsetstrokecolor{currentstroke}%
\pgfsetstrokeopacity{0.000000}%
\pgfsetdash{}{0pt}%
\pgfpathmoveto{\pgfqpoint{0.565124in}{0.549691in}}%
\pgfpathlineto{\pgfqpoint{5.850000in}{0.549691in}}%
\pgfpathlineto{\pgfqpoint{5.850000in}{3.850000in}}%
\pgfpathlineto{\pgfqpoint{0.565124in}{3.850000in}}%
\pgfpathclose%
\pgfusepath{fill}%
\end{pgfscope}%
\begin{pgfscope}%
\pgfsetbuttcap%
\pgfsetroundjoin%
\definecolor{currentfill}{rgb}{0.000000,0.000000,0.000000}%
\pgfsetfillcolor{currentfill}%
\pgfsetlinewidth{0.803000pt}%
\definecolor{currentstroke}{rgb}{0.000000,0.000000,0.000000}%
\pgfsetstrokecolor{currentstroke}%
\pgfsetdash{}{0pt}%
\pgfsys@defobject{currentmarker}{\pgfqpoint{0.000000in}{-0.048611in}}{\pgfqpoint{0.000000in}{0.000000in}}{%
\pgfpathmoveto{\pgfqpoint{0.000000in}{0.000000in}}%
\pgfpathlineto{\pgfqpoint{0.000000in}{-0.048611in}}%
\pgfusepath{stroke,fill}%
}%
\begin{pgfscope}%
\pgfsys@transformshift{1.285789in}{0.549691in}%
\pgfsys@useobject{currentmarker}{}%
\end{pgfscope}%
\end{pgfscope}%
\begin{pgfscope}%
\definecolor{textcolor}{rgb}{0.000000,0.000000,0.000000}%
\pgfsetstrokecolor{textcolor}%
\pgfsetfillcolor{textcolor}%
\pgftext[x=1.285789in,y=0.452469in,,top]{\color{textcolor}\rmfamily\fontsize{10.000000}{12.000000}\selectfont \(\displaystyle {−0.4}\)}%
\end{pgfscope}%
\begin{pgfscope}%
\pgfsetbuttcap%
\pgfsetroundjoin%
\definecolor{currentfill}{rgb}{0.000000,0.000000,0.000000}%
\pgfsetfillcolor{currentfill}%
\pgfsetlinewidth{0.803000pt}%
\definecolor{currentstroke}{rgb}{0.000000,0.000000,0.000000}%
\pgfsetstrokecolor{currentstroke}%
\pgfsetdash{}{0pt}%
\pgfsys@defobject{currentmarker}{\pgfqpoint{0.000000in}{-0.048611in}}{\pgfqpoint{0.000000in}{0.000000in}}{%
\pgfpathmoveto{\pgfqpoint{0.000000in}{0.000000in}}%
\pgfpathlineto{\pgfqpoint{0.000000in}{-0.048611in}}%
\pgfusepath{stroke,fill}%
}%
\begin{pgfscope}%
\pgfsys@transformshift{2.246675in}{0.549691in}%
\pgfsys@useobject{currentmarker}{}%
\end{pgfscope}%
\end{pgfscope}%
\begin{pgfscope}%
\definecolor{textcolor}{rgb}{0.000000,0.000000,0.000000}%
\pgfsetstrokecolor{textcolor}%
\pgfsetfillcolor{textcolor}%
\pgftext[x=2.246675in,y=0.452469in,,top]{\color{textcolor}\rmfamily\fontsize{10.000000}{12.000000}\selectfont \(\displaystyle {−0.2}\)}%
\end{pgfscope}%
\begin{pgfscope}%
\pgfsetbuttcap%
\pgfsetroundjoin%
\definecolor{currentfill}{rgb}{0.000000,0.000000,0.000000}%
\pgfsetfillcolor{currentfill}%
\pgfsetlinewidth{0.803000pt}%
\definecolor{currentstroke}{rgb}{0.000000,0.000000,0.000000}%
\pgfsetstrokecolor{currentstroke}%
\pgfsetdash{}{0pt}%
\pgfsys@defobject{currentmarker}{\pgfqpoint{0.000000in}{-0.048611in}}{\pgfqpoint{0.000000in}{0.000000in}}{%
\pgfpathmoveto{\pgfqpoint{0.000000in}{0.000000in}}%
\pgfpathlineto{\pgfqpoint{0.000000in}{-0.048611in}}%
\pgfusepath{stroke,fill}%
}%
\begin{pgfscope}%
\pgfsys@transformshift{3.207562in}{0.549691in}%
\pgfsys@useobject{currentmarker}{}%
\end{pgfscope}%
\end{pgfscope}%
\begin{pgfscope}%
\definecolor{textcolor}{rgb}{0.000000,0.000000,0.000000}%
\pgfsetstrokecolor{textcolor}%
\pgfsetfillcolor{textcolor}%
\pgftext[x=3.207562in,y=0.452469in,,top]{\color{textcolor}\rmfamily\fontsize{10.000000}{12.000000}\selectfont \(\displaystyle {0.0}\)}%
\end{pgfscope}%
\begin{pgfscope}%
\pgfsetbuttcap%
\pgfsetroundjoin%
\definecolor{currentfill}{rgb}{0.000000,0.000000,0.000000}%
\pgfsetfillcolor{currentfill}%
\pgfsetlinewidth{0.803000pt}%
\definecolor{currentstroke}{rgb}{0.000000,0.000000,0.000000}%
\pgfsetstrokecolor{currentstroke}%
\pgfsetdash{}{0pt}%
\pgfsys@defobject{currentmarker}{\pgfqpoint{0.000000in}{-0.048611in}}{\pgfqpoint{0.000000in}{0.000000in}}{%
\pgfpathmoveto{\pgfqpoint{0.000000in}{0.000000in}}%
\pgfpathlineto{\pgfqpoint{0.000000in}{-0.048611in}}%
\pgfusepath{stroke,fill}%
}%
\begin{pgfscope}%
\pgfsys@transformshift{4.168448in}{0.549691in}%
\pgfsys@useobject{currentmarker}{}%
\end{pgfscope}%
\end{pgfscope}%
\begin{pgfscope}%
\definecolor{textcolor}{rgb}{0.000000,0.000000,0.000000}%
\pgfsetstrokecolor{textcolor}%
\pgfsetfillcolor{textcolor}%
\pgftext[x=4.168448in,y=0.452469in,,top]{\color{textcolor}\rmfamily\fontsize{10.000000}{12.000000}\selectfont \(\displaystyle {0.2}\)}%
\end{pgfscope}%
\begin{pgfscope}%
\pgfsetbuttcap%
\pgfsetroundjoin%
\definecolor{currentfill}{rgb}{0.000000,0.000000,0.000000}%
\pgfsetfillcolor{currentfill}%
\pgfsetlinewidth{0.803000pt}%
\definecolor{currentstroke}{rgb}{0.000000,0.000000,0.000000}%
\pgfsetstrokecolor{currentstroke}%
\pgfsetdash{}{0pt}%
\pgfsys@defobject{currentmarker}{\pgfqpoint{0.000000in}{-0.048611in}}{\pgfqpoint{0.000000in}{0.000000in}}{%
\pgfpathmoveto{\pgfqpoint{0.000000in}{0.000000in}}%
\pgfpathlineto{\pgfqpoint{0.000000in}{-0.048611in}}%
\pgfusepath{stroke,fill}%
}%
\begin{pgfscope}%
\pgfsys@transformshift{5.129335in}{0.549691in}%
\pgfsys@useobject{currentmarker}{}%
\end{pgfscope}%
\end{pgfscope}%
\begin{pgfscope}%
\definecolor{textcolor}{rgb}{0.000000,0.000000,0.000000}%
\pgfsetstrokecolor{textcolor}%
\pgfsetfillcolor{textcolor}%
\pgftext[x=5.129335in,y=0.452469in,,top]{\color{textcolor}\rmfamily\fontsize{10.000000}{12.000000}\selectfont \(\displaystyle {0.4}\)}%
\end{pgfscope}%
\begin{pgfscope}%
\definecolor{textcolor}{rgb}{0.000000,0.000000,0.000000}%
\pgfsetstrokecolor{textcolor}%
\pgfsetfillcolor{textcolor}%
\pgftext[x=3.207562in,y=0.273457in,,top]{\color{textcolor}\rmfamily\fontsize{10.000000}{12.000000}\selectfont \(\displaystyle \Delta\)}%
\end{pgfscope}%
\begin{pgfscope}%
\pgfsetbuttcap%
\pgfsetroundjoin%
\definecolor{currentfill}{rgb}{0.000000,0.000000,0.000000}%
\pgfsetfillcolor{currentfill}%
\pgfsetlinewidth{0.803000pt}%
\definecolor{currentstroke}{rgb}{0.000000,0.000000,0.000000}%
\pgfsetstrokecolor{currentstroke}%
\pgfsetdash{}{0pt}%
\pgfsys@defobject{currentmarker}{\pgfqpoint{-0.048611in}{0.000000in}}{\pgfqpoint{-0.000000in}{0.000000in}}{%
\pgfpathmoveto{\pgfqpoint{-0.000000in}{0.000000in}}%
\pgfpathlineto{\pgfqpoint{-0.048611in}{0.000000in}}%
\pgfusepath{stroke,fill}%
}%
\begin{pgfscope}%
\pgfsys@transformshift{0.565124in}{0.699705in}%
\pgfsys@useobject{currentmarker}{}%
\end{pgfscope}%
\end{pgfscope}%
\begin{pgfscope}%
\definecolor{textcolor}{rgb}{0.000000,0.000000,0.000000}%
\pgfsetstrokecolor{textcolor}%
\pgfsetfillcolor{textcolor}%
\pgftext[x=0.398457in, y=0.651480in, left, base]{\color{textcolor}\rmfamily\fontsize{10.000000}{12.000000}\selectfont \(\displaystyle {0}\)}%
\end{pgfscope}%
\begin{pgfscope}%
\pgfsetbuttcap%
\pgfsetroundjoin%
\definecolor{currentfill}{rgb}{0.000000,0.000000,0.000000}%
\pgfsetfillcolor{currentfill}%
\pgfsetlinewidth{0.803000pt}%
\definecolor{currentstroke}{rgb}{0.000000,0.000000,0.000000}%
\pgfsetstrokecolor{currentstroke}%
\pgfsetdash{}{0pt}%
\pgfsys@defobject{currentmarker}{\pgfqpoint{-0.048611in}{0.000000in}}{\pgfqpoint{-0.000000in}{0.000000in}}{%
\pgfpathmoveto{\pgfqpoint{-0.000000in}{0.000000in}}%
\pgfpathlineto{\pgfqpoint{-0.048611in}{0.000000in}}%
\pgfusepath{stroke,fill}%
}%
\begin{pgfscope}%
\pgfsys@transformshift{0.565124in}{1.277518in}%
\pgfsys@useobject{currentmarker}{}%
\end{pgfscope}%
\end{pgfscope}%
\begin{pgfscope}%
\definecolor{textcolor}{rgb}{0.000000,0.000000,0.000000}%
\pgfsetstrokecolor{textcolor}%
\pgfsetfillcolor{textcolor}%
\pgftext[x=0.329012in, y=1.229293in, left, base]{\color{textcolor}\rmfamily\fontsize{10.000000}{12.000000}\selectfont \(\displaystyle {10}\)}%
\end{pgfscope}%
\begin{pgfscope}%
\pgfsetbuttcap%
\pgfsetroundjoin%
\definecolor{currentfill}{rgb}{0.000000,0.000000,0.000000}%
\pgfsetfillcolor{currentfill}%
\pgfsetlinewidth{0.803000pt}%
\definecolor{currentstroke}{rgb}{0.000000,0.000000,0.000000}%
\pgfsetstrokecolor{currentstroke}%
\pgfsetdash{}{0pt}%
\pgfsys@defobject{currentmarker}{\pgfqpoint{-0.048611in}{0.000000in}}{\pgfqpoint{-0.000000in}{0.000000in}}{%
\pgfpathmoveto{\pgfqpoint{-0.000000in}{0.000000in}}%
\pgfpathlineto{\pgfqpoint{-0.048611in}{0.000000in}}%
\pgfusepath{stroke,fill}%
}%
\begin{pgfscope}%
\pgfsys@transformshift{0.565124in}{1.855331in}%
\pgfsys@useobject{currentmarker}{}%
\end{pgfscope}%
\end{pgfscope}%
\begin{pgfscope}%
\definecolor{textcolor}{rgb}{0.000000,0.000000,0.000000}%
\pgfsetstrokecolor{textcolor}%
\pgfsetfillcolor{textcolor}%
\pgftext[x=0.329012in, y=1.807105in, left, base]{\color{textcolor}\rmfamily\fontsize{10.000000}{12.000000}\selectfont \(\displaystyle {20}\)}%
\end{pgfscope}%
\begin{pgfscope}%
\pgfsetbuttcap%
\pgfsetroundjoin%
\definecolor{currentfill}{rgb}{0.000000,0.000000,0.000000}%
\pgfsetfillcolor{currentfill}%
\pgfsetlinewidth{0.803000pt}%
\definecolor{currentstroke}{rgb}{0.000000,0.000000,0.000000}%
\pgfsetstrokecolor{currentstroke}%
\pgfsetdash{}{0pt}%
\pgfsys@defobject{currentmarker}{\pgfqpoint{-0.048611in}{0.000000in}}{\pgfqpoint{-0.000000in}{0.000000in}}{%
\pgfpathmoveto{\pgfqpoint{-0.000000in}{0.000000in}}%
\pgfpathlineto{\pgfqpoint{-0.048611in}{0.000000in}}%
\pgfusepath{stroke,fill}%
}%
\begin{pgfscope}%
\pgfsys@transformshift{0.565124in}{2.433143in}%
\pgfsys@useobject{currentmarker}{}%
\end{pgfscope}%
\end{pgfscope}%
\begin{pgfscope}%
\definecolor{textcolor}{rgb}{0.000000,0.000000,0.000000}%
\pgfsetstrokecolor{textcolor}%
\pgfsetfillcolor{textcolor}%
\pgftext[x=0.329012in, y=2.384918in, left, base]{\color{textcolor}\rmfamily\fontsize{10.000000}{12.000000}\selectfont \(\displaystyle {30}\)}%
\end{pgfscope}%
\begin{pgfscope}%
\pgfsetbuttcap%
\pgfsetroundjoin%
\definecolor{currentfill}{rgb}{0.000000,0.000000,0.000000}%
\pgfsetfillcolor{currentfill}%
\pgfsetlinewidth{0.803000pt}%
\definecolor{currentstroke}{rgb}{0.000000,0.000000,0.000000}%
\pgfsetstrokecolor{currentstroke}%
\pgfsetdash{}{0pt}%
\pgfsys@defobject{currentmarker}{\pgfqpoint{-0.048611in}{0.000000in}}{\pgfqpoint{-0.000000in}{0.000000in}}{%
\pgfpathmoveto{\pgfqpoint{-0.000000in}{0.000000in}}%
\pgfpathlineto{\pgfqpoint{-0.048611in}{0.000000in}}%
\pgfusepath{stroke,fill}%
}%
\begin{pgfscope}%
\pgfsys@transformshift{0.565124in}{3.010956in}%
\pgfsys@useobject{currentmarker}{}%
\end{pgfscope}%
\end{pgfscope}%
\begin{pgfscope}%
\definecolor{textcolor}{rgb}{0.000000,0.000000,0.000000}%
\pgfsetstrokecolor{textcolor}%
\pgfsetfillcolor{textcolor}%
\pgftext[x=0.329012in, y=2.962731in, left, base]{\color{textcolor}\rmfamily\fontsize{10.000000}{12.000000}\selectfont \(\displaystyle {40}\)}%
\end{pgfscope}%
\begin{pgfscope}%
\pgfsetbuttcap%
\pgfsetroundjoin%
\definecolor{currentfill}{rgb}{0.000000,0.000000,0.000000}%
\pgfsetfillcolor{currentfill}%
\pgfsetlinewidth{0.803000pt}%
\definecolor{currentstroke}{rgb}{0.000000,0.000000,0.000000}%
\pgfsetstrokecolor{currentstroke}%
\pgfsetdash{}{0pt}%
\pgfsys@defobject{currentmarker}{\pgfqpoint{-0.048611in}{0.000000in}}{\pgfqpoint{-0.000000in}{0.000000in}}{%
\pgfpathmoveto{\pgfqpoint{-0.000000in}{0.000000in}}%
\pgfpathlineto{\pgfqpoint{-0.048611in}{0.000000in}}%
\pgfusepath{stroke,fill}%
}%
\begin{pgfscope}%
\pgfsys@transformshift{0.565124in}{3.588769in}%
\pgfsys@useobject{currentmarker}{}%
\end{pgfscope}%
\end{pgfscope}%
\begin{pgfscope}%
\definecolor{textcolor}{rgb}{0.000000,0.000000,0.000000}%
\pgfsetstrokecolor{textcolor}%
\pgfsetfillcolor{textcolor}%
\pgftext[x=0.329012in, y=3.540543in, left, base]{\color{textcolor}\rmfamily\fontsize{10.000000}{12.000000}\selectfont \(\displaystyle {50}\)}%
\end{pgfscope}%
\begin{pgfscope}%
\definecolor{textcolor}{rgb}{0.000000,0.000000,0.000000}%
\pgfsetstrokecolor{textcolor}%
\pgfsetfillcolor{textcolor}%
\pgftext[x=0.273457in,y=2.199846in,,bottom,rotate=90.000000]{\color{textcolor}\rmfamily\fontsize{10.000000}{12.000000}\selectfont Regret}%
\end{pgfscope}%
\begin{pgfscope}%
\pgfpathrectangle{\pgfqpoint{0.565124in}{0.549691in}}{\pgfqpoint{5.284876in}{3.300309in}}%
\pgfusepath{clip}%
\pgfsetrectcap%
\pgfsetroundjoin%
\pgfsetlinewidth{1.505625pt}%
\definecolor{currentstroke}{rgb}{0.121569,0.466667,0.705882}%
\pgfsetstrokecolor{currentstroke}%
\pgfsetdash{}{0pt}%
\pgfpathmoveto{\pgfqpoint{0.805345in}{1.243427in}}%
\pgfpathlineto{\pgfqpoint{0.853390in}{1.247275in}}%
\pgfpathlineto{\pgfqpoint{0.901434in}{1.274929in}}%
\pgfpathlineto{\pgfqpoint{0.949478in}{1.261316in}}%
\pgfpathlineto{\pgfqpoint{0.997523in}{1.276478in}}%
\pgfpathlineto{\pgfqpoint{1.045567in}{1.332583in}}%
\pgfpathlineto{\pgfqpoint{1.093611in}{1.314960in}}%
\pgfpathlineto{\pgfqpoint{1.141656in}{1.319363in}}%
\pgfpathlineto{\pgfqpoint{1.189700in}{1.308835in}}%
\pgfpathlineto{\pgfqpoint{1.237744in}{1.337449in}}%
\pgfpathlineto{\pgfqpoint{1.285789in}{1.351940in}}%
\pgfpathlineto{\pgfqpoint{1.333833in}{1.332029in}}%
\pgfpathlineto{\pgfqpoint{1.381877in}{1.384321in}}%
\pgfpathlineto{\pgfqpoint{1.429922in}{1.432152in}}%
\pgfpathlineto{\pgfqpoint{1.477966in}{1.444390in}}%
\pgfpathlineto{\pgfqpoint{1.526010in}{1.386898in}}%
\pgfpathlineto{\pgfqpoint{1.574055in}{1.528751in}}%
\pgfpathlineto{\pgfqpoint{1.622099in}{1.447164in}}%
\pgfpathlineto{\pgfqpoint{1.670143in}{1.511786in}}%
\pgfpathlineto{\pgfqpoint{1.718188in}{1.513277in}}%
\pgfpathlineto{\pgfqpoint{1.766232in}{1.423936in}}%
\pgfpathlineto{\pgfqpoint{1.814276in}{1.463134in}}%
\pgfpathlineto{\pgfqpoint{1.862321in}{1.614128in}}%
\pgfpathlineto{\pgfqpoint{1.910365in}{1.645122in}}%
\pgfpathlineto{\pgfqpoint{1.958409in}{1.585168in}}%
\pgfpathlineto{\pgfqpoint{2.006454in}{1.584625in}}%
\pgfpathlineto{\pgfqpoint{2.054498in}{1.671540in}}%
\pgfpathlineto{\pgfqpoint{2.102542in}{1.788131in}}%
\pgfpathlineto{\pgfqpoint{2.150587in}{1.714113in}}%
\pgfpathlineto{\pgfqpoint{2.198631in}{1.831086in}}%
\pgfpathlineto{\pgfqpoint{2.246675in}{1.698628in}}%
\pgfpathlineto{\pgfqpoint{2.294720in}{1.990550in}}%
\pgfpathlineto{\pgfqpoint{2.342764in}{1.876640in}}%
\pgfpathlineto{\pgfqpoint{2.390808in}{2.148963in}}%
\pgfpathlineto{\pgfqpoint{2.438853in}{2.065932in}}%
\pgfpathlineto{\pgfqpoint{2.486897in}{2.164287in}}%
\pgfpathlineto{\pgfqpoint{2.534941in}{2.198829in}}%
\pgfpathlineto{\pgfqpoint{2.582986in}{2.276683in}}%
\pgfpathlineto{\pgfqpoint{2.631030in}{2.311248in}}%
\pgfpathlineto{\pgfqpoint{2.679074in}{2.460936in}}%
\pgfpathlineto{\pgfqpoint{2.727119in}{2.498436in}}%
\pgfpathlineto{\pgfqpoint{2.775163in}{2.568178in}}%
\pgfpathlineto{\pgfqpoint{2.823207in}{2.819399in}}%
\pgfpathlineto{\pgfqpoint{2.871252in}{2.893059in}}%
\pgfpathlineto{\pgfqpoint{2.919296in}{3.072158in}}%
\pgfpathlineto{\pgfqpoint{2.967340in}{3.191291in}}%
\pgfpathlineto{\pgfqpoint{3.015385in}{3.699986in}}%
\pgfpathlineto{\pgfqpoint{3.063429in}{3.262455in}}%
\pgfpathlineto{\pgfqpoint{3.111473in}{3.638715in}}%
\pgfpathlineto{\pgfqpoint{3.159518in}{2.576441in}}%
\pgfpathlineto{\pgfqpoint{3.207562in}{0.699705in}}%
\pgfpathlineto{\pgfqpoint{3.255606in}{2.778513in}}%
\pgfpathlineto{\pgfqpoint{3.303651in}{3.367559in}}%
\pgfpathlineto{\pgfqpoint{3.351695in}{3.589658in}}%
\pgfpathlineto{\pgfqpoint{3.399739in}{3.557983in}}%
\pgfpathlineto{\pgfqpoint{3.447784in}{3.617717in}}%
\pgfpathlineto{\pgfqpoint{3.495828in}{3.202582in}}%
\pgfpathlineto{\pgfqpoint{3.543872in}{3.129754in}}%
\pgfpathlineto{\pgfqpoint{3.591917in}{2.659461in}}%
\pgfpathlineto{\pgfqpoint{3.639961in}{2.790959in}}%
\pgfpathlineto{\pgfqpoint{3.688005in}{2.771164in}}%
\pgfpathlineto{\pgfqpoint{3.736050in}{2.325554in}}%
\pgfpathlineto{\pgfqpoint{3.784094in}{2.209044in}}%
\pgfpathlineto{\pgfqpoint{3.832138in}{2.237923in}}%
\pgfpathlineto{\pgfqpoint{3.880182in}{2.169383in}}%
\pgfpathlineto{\pgfqpoint{3.928227in}{2.007411in}}%
\pgfpathlineto{\pgfqpoint{3.976271in}{1.967750in}}%
\pgfpathlineto{\pgfqpoint{4.024315in}{1.902804in}}%
\pgfpathlineto{\pgfqpoint{4.072360in}{2.019129in}}%
\pgfpathlineto{\pgfqpoint{4.120404in}{1.823678in}}%
\pgfpathlineto{\pgfqpoint{4.168448in}{1.891617in}}%
\pgfpathlineto{\pgfqpoint{4.216493in}{1.675527in}}%
\pgfpathlineto{\pgfqpoint{4.264537in}{1.600723in}}%
\pgfpathlineto{\pgfqpoint{4.312581in}{1.714240in}}%
\pgfpathlineto{\pgfqpoint{4.360626in}{1.651571in}}%
\pgfpathlineto{\pgfqpoint{4.408670in}{1.577692in}}%
\pgfpathlineto{\pgfqpoint{4.456714in}{1.593281in}}%
\pgfpathlineto{\pgfqpoint{4.504759in}{1.590831in}}%
\pgfpathlineto{\pgfqpoint{4.552803in}{1.708936in}}%
\pgfpathlineto{\pgfqpoint{4.600847in}{1.454086in}}%
\pgfpathlineto{\pgfqpoint{4.648892in}{1.511301in}}%
\pgfpathlineto{\pgfqpoint{4.696936in}{1.422283in}}%
\pgfpathlineto{\pgfqpoint{4.744980in}{1.522880in}}%
\pgfpathlineto{\pgfqpoint{4.793025in}{1.435342in}}%
\pgfpathlineto{\pgfqpoint{4.841069in}{1.361370in}}%
\pgfpathlineto{\pgfqpoint{4.889113in}{1.465769in}}%
\pgfpathlineto{\pgfqpoint{4.937158in}{1.422757in}}%
\pgfpathlineto{\pgfqpoint{4.985202in}{1.399656in}}%
\pgfpathlineto{\pgfqpoint{5.033246in}{1.332063in}}%
\pgfpathlineto{\pgfqpoint{5.081291in}{1.357268in}}%
\pgfpathlineto{\pgfqpoint{5.129335in}{1.287225in}}%
\pgfpathlineto{\pgfqpoint{5.177379in}{1.329868in}}%
\pgfpathlineto{\pgfqpoint{5.225424in}{1.378728in}}%
\pgfpathlineto{\pgfqpoint{5.273468in}{1.282094in}}%
\pgfpathlineto{\pgfqpoint{5.321512in}{1.261062in}}%
\pgfpathlineto{\pgfqpoint{5.369557in}{1.302422in}}%
\pgfpathlineto{\pgfqpoint{5.417601in}{1.296147in}}%
\pgfpathlineto{\pgfqpoint{5.465645in}{1.289559in}}%
\pgfpathlineto{\pgfqpoint{5.513690in}{1.287133in}}%
\pgfpathlineto{\pgfqpoint{5.561734in}{1.216131in}}%
\pgfpathlineto{\pgfqpoint{5.609778in}{1.225515in}}%
\pgfusepath{stroke}%
\end{pgfscope}%
\begin{pgfscope}%
\pgfpathrectangle{\pgfqpoint{0.565124in}{0.549691in}}{\pgfqpoint{5.284876in}{3.300309in}}%
\pgfusepath{clip}%
\pgfsetrectcap%
\pgfsetroundjoin%
\pgfsetlinewidth{1.505625pt}%
\definecolor{currentstroke}{rgb}{1.000000,0.498039,0.054902}%
\pgfsetstrokecolor{currentstroke}%
\pgfsetdash{}{0pt}%
\pgfpathmoveto{\pgfqpoint{0.805345in}{1.108219in}}%
\pgfpathlineto{\pgfqpoint{0.853390in}{1.115337in}}%
\pgfpathlineto{\pgfqpoint{0.901434in}{1.141801in}}%
\pgfpathlineto{\pgfqpoint{0.949478in}{1.136936in}}%
\pgfpathlineto{\pgfqpoint{0.997523in}{1.165908in}}%
\pgfpathlineto{\pgfqpoint{1.045567in}{1.180214in}}%
\pgfpathlineto{\pgfqpoint{1.093611in}{1.153265in}}%
\pgfpathlineto{\pgfqpoint{1.141656in}{1.219482in}}%
\pgfpathlineto{\pgfqpoint{1.189700in}{1.196231in}}%
\pgfpathlineto{\pgfqpoint{1.237744in}{1.252637in}}%
\pgfpathlineto{\pgfqpoint{1.285789in}{1.300168in}}%
\pgfpathlineto{\pgfqpoint{1.333833in}{1.352310in}}%
\pgfpathlineto{\pgfqpoint{1.381877in}{1.365877in}}%
\pgfpathlineto{\pgfqpoint{1.429922in}{1.292761in}}%
\pgfpathlineto{\pgfqpoint{1.477966in}{1.364929in}}%
\pgfpathlineto{\pgfqpoint{1.526010in}{1.390942in}}%
\pgfpathlineto{\pgfqpoint{1.574055in}{1.337402in}}%
\pgfpathlineto{\pgfqpoint{1.622099in}{1.399113in}}%
\pgfpathlineto{\pgfqpoint{1.670143in}{1.381247in}}%
\pgfpathlineto{\pgfqpoint{1.718188in}{1.385742in}}%
\pgfpathlineto{\pgfqpoint{1.766232in}{1.430869in}}%
\pgfpathlineto{\pgfqpoint{1.814276in}{1.393427in}}%
\pgfpathlineto{\pgfqpoint{1.862321in}{1.435839in}}%
\pgfpathlineto{\pgfqpoint{1.910365in}{1.537476in}}%
\pgfpathlineto{\pgfqpoint{1.958409in}{1.550615in}}%
\pgfpathlineto{\pgfqpoint{2.006454in}{1.517888in}}%
\pgfpathlineto{\pgfqpoint{2.054498in}{1.485161in}}%
\pgfpathlineto{\pgfqpoint{2.102542in}{1.562206in}}%
\pgfpathlineto{\pgfqpoint{2.150587in}{1.626147in}}%
\pgfpathlineto{\pgfqpoint{2.198631in}{1.610731in}}%
\pgfpathlineto{\pgfqpoint{2.246675in}{1.728443in}}%
\pgfpathlineto{\pgfqpoint{2.294720in}{1.784814in}}%
\pgfpathlineto{\pgfqpoint{2.342764in}{1.819437in}}%
\pgfpathlineto{\pgfqpoint{2.390808in}{1.793967in}}%
\pgfpathlineto{\pgfqpoint{2.438853in}{1.862172in}}%
\pgfpathlineto{\pgfqpoint{2.486897in}{2.131698in}}%
\pgfpathlineto{\pgfqpoint{2.534941in}{1.855839in}}%
\pgfpathlineto{\pgfqpoint{2.582986in}{2.040820in}}%
\pgfpathlineto{\pgfqpoint{2.631030in}{1.933636in}}%
\pgfpathlineto{\pgfqpoint{2.679074in}{2.225385in}}%
\pgfpathlineto{\pgfqpoint{2.727119in}{2.202943in}}%
\pgfpathlineto{\pgfqpoint{2.775163in}{2.346749in}}%
\pgfpathlineto{\pgfqpoint{2.823207in}{3.003005in}}%
\pgfpathlineto{\pgfqpoint{2.871252in}{3.235887in}}%
\pgfpathlineto{\pgfqpoint{2.919296in}{2.983614in}}%
\pgfpathlineto{\pgfqpoint{2.967340in}{3.311014in}}%
\pgfpathlineto{\pgfqpoint{3.015385in}{3.034762in}}%
\pgfpathlineto{\pgfqpoint{3.063429in}{3.039604in}}%
\pgfpathlineto{\pgfqpoint{3.111473in}{3.212936in}}%
\pgfpathlineto{\pgfqpoint{3.159518in}{2.523213in}}%
\pgfpathlineto{\pgfqpoint{3.207562in}{0.699705in}}%
\pgfpathlineto{\pgfqpoint{3.255606in}{2.243309in}}%
\pgfpathlineto{\pgfqpoint{3.303651in}{3.092497in}}%
\pgfpathlineto{\pgfqpoint{3.351695in}{2.775775in}}%
\pgfpathlineto{\pgfqpoint{3.399739in}{3.249015in}}%
\pgfpathlineto{\pgfqpoint{3.447784in}{3.496030in}}%
\pgfpathlineto{\pgfqpoint{3.495828in}{3.004068in}}%
\pgfpathlineto{\pgfqpoint{3.543872in}{3.087528in}}%
\pgfpathlineto{\pgfqpoint{3.591917in}{2.405870in}}%
\pgfpathlineto{\pgfqpoint{3.639961in}{2.432554in}}%
\pgfpathlineto{\pgfqpoint{3.688005in}{2.110839in}}%
\pgfpathlineto{\pgfqpoint{3.736050in}{2.176444in}}%
\pgfpathlineto{\pgfqpoint{3.784094in}{2.224576in}}%
\pgfpathlineto{\pgfqpoint{3.832138in}{2.158601in}}%
\pgfpathlineto{\pgfqpoint{3.880182in}{2.088651in}}%
\pgfpathlineto{\pgfqpoint{3.928227in}{2.028212in}}%
\pgfpathlineto{\pgfqpoint{3.976271in}{1.891941in}}%
\pgfpathlineto{\pgfqpoint{4.024315in}{1.888855in}}%
\pgfpathlineto{\pgfqpoint{4.072360in}{1.727079in}}%
\pgfpathlineto{\pgfqpoint{4.120404in}{1.662075in}}%
\pgfpathlineto{\pgfqpoint{4.168448in}{1.637148in}}%
\pgfpathlineto{\pgfqpoint{4.216493in}{1.782792in}}%
\pgfpathlineto{\pgfqpoint{4.264537in}{1.686401in}}%
\pgfpathlineto{\pgfqpoint{4.312581in}{1.528982in}}%
\pgfpathlineto{\pgfqpoint{4.360626in}{1.494036in}}%
\pgfpathlineto{\pgfqpoint{4.408670in}{1.452017in}}%
\pgfpathlineto{\pgfqpoint{4.456714in}{1.410600in}}%
\pgfpathlineto{\pgfqpoint{4.504759in}{1.469144in}}%
\pgfpathlineto{\pgfqpoint{4.552803in}{1.301231in}}%
\pgfpathlineto{\pgfqpoint{4.600847in}{1.363600in}}%
\pgfpathlineto{\pgfqpoint{4.648892in}{1.249898in}}%
\pgfpathlineto{\pgfqpoint{4.696936in}{1.323766in}}%
\pgfpathlineto{\pgfqpoint{4.744980in}{1.178596in}}%
\pgfpathlineto{\pgfqpoint{4.793025in}{1.216062in}}%
\pgfpathlineto{\pgfqpoint{4.841069in}{1.108334in}}%
\pgfpathlineto{\pgfqpoint{4.889113in}{1.162418in}}%
\pgfpathlineto{\pgfqpoint{4.937158in}{1.133619in}}%
\pgfpathlineto{\pgfqpoint{4.985202in}{1.108045in}}%
\pgfpathlineto{\pgfqpoint{5.033246in}{1.041793in}}%
\pgfpathlineto{\pgfqpoint{5.081291in}{1.044486in}}%
\pgfpathlineto{\pgfqpoint{5.129335in}{1.043619in}}%
\pgfpathlineto{\pgfqpoint{5.177379in}{1.011470in}}%
\pgfpathlineto{\pgfqpoint{5.225424in}{0.956947in}}%
\pgfpathlineto{\pgfqpoint{5.273468in}{0.879590in}}%
\pgfpathlineto{\pgfqpoint{5.321512in}{0.851231in}}%
\pgfpathlineto{\pgfqpoint{5.369557in}{0.813592in}}%
\pgfpathlineto{\pgfqpoint{5.417601in}{0.802302in}}%
\pgfpathlineto{\pgfqpoint{5.465645in}{0.782806in}}%
\pgfpathlineto{\pgfqpoint{5.513690in}{0.775144in}}%
\pgfpathlineto{\pgfqpoint{5.561734in}{0.778981in}}%
\pgfpathlineto{\pgfqpoint{5.609778in}{0.761531in}}%
\pgfusepath{stroke}%
\end{pgfscope}%
\begin{pgfscope}%
\pgfsetrectcap%
\pgfsetmiterjoin%
\pgfsetlinewidth{0.803000pt}%
\definecolor{currentstroke}{rgb}{0.000000,0.000000,0.000000}%
\pgfsetstrokecolor{currentstroke}%
\pgfsetdash{}{0pt}%
\pgfpathmoveto{\pgfqpoint{0.565124in}{0.549691in}}%
\pgfpathlineto{\pgfqpoint{0.565124in}{3.850000in}}%
\pgfusepath{stroke}%
\end{pgfscope}%
\begin{pgfscope}%
\pgfsetrectcap%
\pgfsetmiterjoin%
\pgfsetlinewidth{0.803000pt}%
\definecolor{currentstroke}{rgb}{0.000000,0.000000,0.000000}%
\pgfsetstrokecolor{currentstroke}%
\pgfsetdash{}{0pt}%
\pgfpathmoveto{\pgfqpoint{5.850000in}{0.549691in}}%
\pgfpathlineto{\pgfqpoint{5.850000in}{3.850000in}}%
\pgfusepath{stroke}%
\end{pgfscope}%
\begin{pgfscope}%
\pgfsetrectcap%
\pgfsetmiterjoin%
\pgfsetlinewidth{0.803000pt}%
\definecolor{currentstroke}{rgb}{0.000000,0.000000,0.000000}%
\pgfsetstrokecolor{currentstroke}%
\pgfsetdash{}{0pt}%
\pgfpathmoveto{\pgfqpoint{0.565124in}{0.549691in}}%
\pgfpathlineto{\pgfqpoint{5.850000in}{0.549691in}}%
\pgfusepath{stroke}%
\end{pgfscope}%
\begin{pgfscope}%
\pgfsetrectcap%
\pgfsetmiterjoin%
\pgfsetlinewidth{0.803000pt}%
\definecolor{currentstroke}{rgb}{0.000000,0.000000,0.000000}%
\pgfsetstrokecolor{currentstroke}%
\pgfsetdash{}{0pt}%
\pgfpathmoveto{\pgfqpoint{0.565124in}{3.850000in}}%
\pgfpathlineto{\pgfqpoint{5.850000in}{3.850000in}}%
\pgfusepath{stroke}%
\end{pgfscope}%
\begin{pgfscope}%
\pgfsetbuttcap%
\pgfsetmiterjoin%
\definecolor{currentfill}{rgb}{1.000000,1.000000,1.000000}%
\pgfsetfillcolor{currentfill}%
\pgfsetfillopacity{0.800000}%
\pgfsetlinewidth{1.003750pt}%
\definecolor{currentstroke}{rgb}{0.800000,0.800000,0.800000}%
\pgfsetstrokecolor{currentstroke}%
\pgfsetstrokeopacity{0.800000}%
\pgfsetdash{}{0pt}%
\pgfpathmoveto{\pgfqpoint{4.764351in}{3.351543in}}%
\pgfpathlineto{\pgfqpoint{5.752778in}{3.351543in}}%
\pgfpathquadraticcurveto{\pgfqpoint{5.780556in}{3.351543in}}{\pgfqpoint{5.780556in}{3.379321in}}%
\pgfpathlineto{\pgfqpoint{5.780556in}{3.752778in}}%
\pgfpathquadraticcurveto{\pgfqpoint{5.780556in}{3.780556in}}{\pgfqpoint{5.752778in}{3.780556in}}%
\pgfpathlineto{\pgfqpoint{4.764351in}{3.780556in}}%
\pgfpathquadraticcurveto{\pgfqpoint{4.736573in}{3.780556in}}{\pgfqpoint{4.736573in}{3.752778in}}%
\pgfpathlineto{\pgfqpoint{4.736573in}{3.379321in}}%
\pgfpathquadraticcurveto{\pgfqpoint{4.736573in}{3.351543in}}{\pgfqpoint{4.764351in}{3.351543in}}%
\pgfpathclose%
\pgfusepath{stroke,fill}%
\end{pgfscope}%
\begin{pgfscope}%
\pgfsetrectcap%
\pgfsetroundjoin%
\pgfsetlinewidth{1.505625pt}%
\definecolor{currentstroke}{rgb}{0.121569,0.466667,0.705882}%
\pgfsetstrokecolor{currentstroke}%
\pgfsetdash{}{0pt}%
\pgfpathmoveto{\pgfqpoint{4.792128in}{3.676389in}}%
\pgfpathlineto{\pgfqpoint{5.069906in}{3.676389in}}%
\pgfusepath{stroke}%
\end{pgfscope}%
\begin{pgfscope}%
\definecolor{textcolor}{rgb}{0.000000,0.000000,0.000000}%
\pgfsetstrokecolor{textcolor}%
\pgfsetfillcolor{textcolor}%
\pgftext[x=5.181017in,y=3.627778in,left,base]{\color{textcolor}\rmfamily\fontsize{10.000000}{12.000000}\selectfont UCB}%
\end{pgfscope}%
\begin{pgfscope}%
\pgfsetrectcap%
\pgfsetroundjoin%
\pgfsetlinewidth{1.505625pt}%
\definecolor{currentstroke}{rgb}{1.000000,0.498039,0.054902}%
\pgfsetstrokecolor{currentstroke}%
\pgfsetdash{}{0pt}%
\pgfpathmoveto{\pgfqpoint{4.792128in}{3.482716in}}%
\pgfpathlineto{\pgfqpoint{5.069906in}{3.482716in}}%
\pgfusepath{stroke}%
\end{pgfscope}%
\begin{pgfscope}%
\definecolor{textcolor}{rgb}{0.000000,0.000000,0.000000}%
\pgfsetstrokecolor{textcolor}%
\pgfsetfillcolor{textcolor}%
\pgftext[x=5.181017in,y=3.434105in,left,base]{\color{textcolor}\rmfamily\fontsize{10.000000}{12.000000}\selectfont KL-UCB}%
\end{pgfscope}%
\end{pgfpicture}%
\makeatother%
\endgroup%

		\end{center}
		\caption{$\mu_1 = 0.5$}
	\end{subfigure}
	\bigskip
	\begin{subfigure}[h]{0.49\textwidth}
		\begin{center}
			%% Creator: Matplotlib, PGF backend
%%
%% To include the figure in your LaTeX document, write
%%   \input{<filename>.pgf}
%%
%% Make sure the required packages are loaded in your preamble
%%   \usepackage{pgf}
%%
%% and, on pdftex
%%   \usepackage[utf8]{inputenc}\DeclareUnicodeCharacter{2212}{-}
%%
%% or, on luatex and xetex
%%   \usepackage{unicode-math}
%%
%% Figures using additional raster images can only be included by \input if
%% they are in the same directory as the main LaTeX file. For loading figures
%% from other directories you can use the `import` package
%%   \usepackage{import}
%%
%% and then include the figures with
%%   \import{<path to file>}{<filename>.pgf}
%%
%% Matplotlib used the following preamble
%%
\begingroup%
\makeatletter%
\begin{pgfpicture}%
\pgfpathrectangle{\pgfpointorigin}{\pgfqpoint{3.200000in}{4.000000in}}%
\pgfusepath{use as bounding box, clip}%
\begin{pgfscope}%
\pgfsetbuttcap%
\pgfsetmiterjoin%
\pgfsetlinewidth{0.000000pt}%
\definecolor{currentstroke}{rgb}{1.000000,1.000000,1.000000}%
\pgfsetstrokecolor{currentstroke}%
\pgfsetstrokeopacity{0.000000}%
\pgfsetdash{}{0pt}%
\pgfpathmoveto{\pgfqpoint{0.000000in}{0.000000in}}%
\pgfpathlineto{\pgfqpoint{3.200000in}{0.000000in}}%
\pgfpathlineto{\pgfqpoint{3.200000in}{4.000000in}}%
\pgfpathlineto{\pgfqpoint{0.000000in}{4.000000in}}%
\pgfpathclose%
\pgfusepath{}%
\end{pgfscope}%
\begin{pgfscope}%
\pgfsetbuttcap%
\pgfsetmiterjoin%
\definecolor{currentfill}{rgb}{1.000000,1.000000,1.000000}%
\pgfsetfillcolor{currentfill}%
\pgfsetlinewidth{0.000000pt}%
\definecolor{currentstroke}{rgb}{0.000000,0.000000,0.000000}%
\pgfsetstrokecolor{currentstroke}%
\pgfsetstrokeopacity{0.000000}%
\pgfsetdash{}{0pt}%
\pgfpathmoveto{\pgfqpoint{0.565124in}{0.549691in}}%
\pgfpathlineto{\pgfqpoint{3.039270in}{0.549691in}}%
\pgfpathlineto{\pgfqpoint{3.039270in}{3.850000in}}%
\pgfpathlineto{\pgfqpoint{0.565124in}{3.850000in}}%
\pgfpathclose%
\pgfusepath{fill}%
\end{pgfscope}%
\begin{pgfscope}%
\pgfsetbuttcap%
\pgfsetroundjoin%
\definecolor{currentfill}{rgb}{0.000000,0.000000,0.000000}%
\pgfsetfillcolor{currentfill}%
\pgfsetlinewidth{0.803000pt}%
\definecolor{currentstroke}{rgb}{0.000000,0.000000,0.000000}%
\pgfsetstrokecolor{currentstroke}%
\pgfsetdash{}{0pt}%
\pgfsys@defobject{currentmarker}{\pgfqpoint{0.000000in}{-0.048611in}}{\pgfqpoint{0.000000in}{0.000000in}}{%
\pgfpathmoveto{\pgfqpoint{0.000000in}{0.000000in}}%
\pgfpathlineto{\pgfqpoint{0.000000in}{-0.048611in}}%
\pgfusepath{stroke,fill}%
}%
\begin{pgfscope}%
\pgfsys@transformshift{0.677585in}{0.549691in}%
\pgfsys@useobject{currentmarker}{}%
\end{pgfscope}%
\end{pgfscope}%
\begin{pgfscope}%
\definecolor{textcolor}{rgb}{0.000000,0.000000,0.000000}%
\pgfsetstrokecolor{textcolor}%
\pgfsetfillcolor{textcolor}%
\pgftext[x=0.677585in,y=0.452469in,,top]{\color{textcolor}\rmfamily\fontsize{10.000000}{12.000000}\selectfont \(\displaystyle {−0.50}\)}%
\end{pgfscope}%
\begin{pgfscope}%
\pgfsetbuttcap%
\pgfsetroundjoin%
\definecolor{currentfill}{rgb}{0.000000,0.000000,0.000000}%
\pgfsetfillcolor{currentfill}%
\pgfsetlinewidth{0.803000pt}%
\definecolor{currentstroke}{rgb}{0.000000,0.000000,0.000000}%
\pgfsetstrokecolor{currentstroke}%
\pgfsetdash{}{0pt}%
\pgfsys@defobject{currentmarker}{\pgfqpoint{0.000000in}{-0.048611in}}{\pgfqpoint{0.000000in}{0.000000in}}{%
\pgfpathmoveto{\pgfqpoint{0.000000in}{0.000000in}}%
\pgfpathlineto{\pgfqpoint{0.000000in}{-0.048611in}}%
\pgfusepath{stroke,fill}%
}%
\begin{pgfscope}%
\pgfsys@transformshift{1.239891in}{0.549691in}%
\pgfsys@useobject{currentmarker}{}%
\end{pgfscope}%
\end{pgfscope}%
\begin{pgfscope}%
\definecolor{textcolor}{rgb}{0.000000,0.000000,0.000000}%
\pgfsetstrokecolor{textcolor}%
\pgfsetfillcolor{textcolor}%
\pgftext[x=1.239891in,y=0.452469in,,top]{\color{textcolor}\rmfamily\fontsize{10.000000}{12.000000}\selectfont \(\displaystyle {−0.25}\)}%
\end{pgfscope}%
\begin{pgfscope}%
\pgfsetbuttcap%
\pgfsetroundjoin%
\definecolor{currentfill}{rgb}{0.000000,0.000000,0.000000}%
\pgfsetfillcolor{currentfill}%
\pgfsetlinewidth{0.803000pt}%
\definecolor{currentstroke}{rgb}{0.000000,0.000000,0.000000}%
\pgfsetstrokecolor{currentstroke}%
\pgfsetdash{}{0pt}%
\pgfsys@defobject{currentmarker}{\pgfqpoint{0.000000in}{-0.048611in}}{\pgfqpoint{0.000000in}{0.000000in}}{%
\pgfpathmoveto{\pgfqpoint{0.000000in}{0.000000in}}%
\pgfpathlineto{\pgfqpoint{0.000000in}{-0.048611in}}%
\pgfusepath{stroke,fill}%
}%
\begin{pgfscope}%
\pgfsys@transformshift{1.802197in}{0.549691in}%
\pgfsys@useobject{currentmarker}{}%
\end{pgfscope}%
\end{pgfscope}%
\begin{pgfscope}%
\definecolor{textcolor}{rgb}{0.000000,0.000000,0.000000}%
\pgfsetstrokecolor{textcolor}%
\pgfsetfillcolor{textcolor}%
\pgftext[x=1.802197in,y=0.452469in,,top]{\color{textcolor}\rmfamily\fontsize{10.000000}{12.000000}\selectfont \(\displaystyle {0.00}\)}%
\end{pgfscope}%
\begin{pgfscope}%
\pgfsetbuttcap%
\pgfsetroundjoin%
\definecolor{currentfill}{rgb}{0.000000,0.000000,0.000000}%
\pgfsetfillcolor{currentfill}%
\pgfsetlinewidth{0.803000pt}%
\definecolor{currentstroke}{rgb}{0.000000,0.000000,0.000000}%
\pgfsetstrokecolor{currentstroke}%
\pgfsetdash{}{0pt}%
\pgfsys@defobject{currentmarker}{\pgfqpoint{0.000000in}{-0.048611in}}{\pgfqpoint{0.000000in}{0.000000in}}{%
\pgfpathmoveto{\pgfqpoint{0.000000in}{0.000000in}}%
\pgfpathlineto{\pgfqpoint{0.000000in}{-0.048611in}}%
\pgfusepath{stroke,fill}%
}%
\begin{pgfscope}%
\pgfsys@transformshift{2.364503in}{0.549691in}%
\pgfsys@useobject{currentmarker}{}%
\end{pgfscope}%
\end{pgfscope}%
\begin{pgfscope}%
\definecolor{textcolor}{rgb}{0.000000,0.000000,0.000000}%
\pgfsetstrokecolor{textcolor}%
\pgfsetfillcolor{textcolor}%
\pgftext[x=2.364503in,y=0.452469in,,top]{\color{textcolor}\rmfamily\fontsize{10.000000}{12.000000}\selectfont \(\displaystyle {0.25}\)}%
\end{pgfscope}%
\begin{pgfscope}%
\pgfsetbuttcap%
\pgfsetroundjoin%
\definecolor{currentfill}{rgb}{0.000000,0.000000,0.000000}%
\pgfsetfillcolor{currentfill}%
\pgfsetlinewidth{0.803000pt}%
\definecolor{currentstroke}{rgb}{0.000000,0.000000,0.000000}%
\pgfsetstrokecolor{currentstroke}%
\pgfsetdash{}{0pt}%
\pgfsys@defobject{currentmarker}{\pgfqpoint{0.000000in}{-0.048611in}}{\pgfqpoint{0.000000in}{0.000000in}}{%
\pgfpathmoveto{\pgfqpoint{0.000000in}{0.000000in}}%
\pgfpathlineto{\pgfqpoint{0.000000in}{-0.048611in}}%
\pgfusepath{stroke,fill}%
}%
\begin{pgfscope}%
\pgfsys@transformshift{2.926809in}{0.549691in}%
\pgfsys@useobject{currentmarker}{}%
\end{pgfscope}%
\end{pgfscope}%
\begin{pgfscope}%
\definecolor{textcolor}{rgb}{0.000000,0.000000,0.000000}%
\pgfsetstrokecolor{textcolor}%
\pgfsetfillcolor{textcolor}%
\pgftext[x=2.926809in,y=0.452469in,,top]{\color{textcolor}\rmfamily\fontsize{10.000000}{12.000000}\selectfont \(\displaystyle {0.50}\)}%
\end{pgfscope}%
\begin{pgfscope}%
\definecolor{textcolor}{rgb}{0.000000,0.000000,0.000000}%
\pgfsetstrokecolor{textcolor}%
\pgfsetfillcolor{textcolor}%
\pgftext[x=1.802197in,y=0.273457in,,top]{\color{textcolor}\rmfamily\fontsize{10.000000}{12.000000}\selectfont \(\displaystyle \Delta\)}%
\end{pgfscope}%
\begin{pgfscope}%
\pgfsetbuttcap%
\pgfsetroundjoin%
\definecolor{currentfill}{rgb}{0.000000,0.000000,0.000000}%
\pgfsetfillcolor{currentfill}%
\pgfsetlinewidth{0.803000pt}%
\definecolor{currentstroke}{rgb}{0.000000,0.000000,0.000000}%
\pgfsetstrokecolor{currentstroke}%
\pgfsetdash{}{0pt}%
\pgfsys@defobject{currentmarker}{\pgfqpoint{-0.048611in}{0.000000in}}{\pgfqpoint{-0.000000in}{0.000000in}}{%
\pgfpathmoveto{\pgfqpoint{-0.000000in}{0.000000in}}%
\pgfpathlineto{\pgfqpoint{-0.048611in}{0.000000in}}%
\pgfusepath{stroke,fill}%
}%
\begin{pgfscope}%
\pgfsys@transformshift{0.565124in}{0.699705in}%
\pgfsys@useobject{currentmarker}{}%
\end{pgfscope}%
\end{pgfscope}%
\begin{pgfscope}%
\definecolor{textcolor}{rgb}{0.000000,0.000000,0.000000}%
\pgfsetstrokecolor{textcolor}%
\pgfsetfillcolor{textcolor}%
\pgftext[x=0.398457in, y=0.651480in, left, base]{\color{textcolor}\rmfamily\fontsize{10.000000}{12.000000}\selectfont \(\displaystyle {0}\)}%
\end{pgfscope}%
\begin{pgfscope}%
\pgfsetbuttcap%
\pgfsetroundjoin%
\definecolor{currentfill}{rgb}{0.000000,0.000000,0.000000}%
\pgfsetfillcolor{currentfill}%
\pgfsetlinewidth{0.803000pt}%
\definecolor{currentstroke}{rgb}{0.000000,0.000000,0.000000}%
\pgfsetstrokecolor{currentstroke}%
\pgfsetdash{}{0pt}%
\pgfsys@defobject{currentmarker}{\pgfqpoint{-0.048611in}{0.000000in}}{\pgfqpoint{-0.000000in}{0.000000in}}{%
\pgfpathmoveto{\pgfqpoint{-0.000000in}{0.000000in}}%
\pgfpathlineto{\pgfqpoint{-0.048611in}{0.000000in}}%
\pgfusepath{stroke,fill}%
}%
\begin{pgfscope}%
\pgfsys@transformshift{0.565124in}{1.274392in}%
\pgfsys@useobject{currentmarker}{}%
\end{pgfscope}%
\end{pgfscope}%
\begin{pgfscope}%
\definecolor{textcolor}{rgb}{0.000000,0.000000,0.000000}%
\pgfsetstrokecolor{textcolor}%
\pgfsetfillcolor{textcolor}%
\pgftext[x=0.329012in, y=1.226167in, left, base]{\color{textcolor}\rmfamily\fontsize{10.000000}{12.000000}\selectfont \(\displaystyle {10}\)}%
\end{pgfscope}%
\begin{pgfscope}%
\pgfsetbuttcap%
\pgfsetroundjoin%
\definecolor{currentfill}{rgb}{0.000000,0.000000,0.000000}%
\pgfsetfillcolor{currentfill}%
\pgfsetlinewidth{0.803000pt}%
\definecolor{currentstroke}{rgb}{0.000000,0.000000,0.000000}%
\pgfsetstrokecolor{currentstroke}%
\pgfsetdash{}{0pt}%
\pgfsys@defobject{currentmarker}{\pgfqpoint{-0.048611in}{0.000000in}}{\pgfqpoint{-0.000000in}{0.000000in}}{%
\pgfpathmoveto{\pgfqpoint{-0.000000in}{0.000000in}}%
\pgfpathlineto{\pgfqpoint{-0.048611in}{0.000000in}}%
\pgfusepath{stroke,fill}%
}%
\begin{pgfscope}%
\pgfsys@transformshift{0.565124in}{1.849079in}%
\pgfsys@useobject{currentmarker}{}%
\end{pgfscope}%
\end{pgfscope}%
\begin{pgfscope}%
\definecolor{textcolor}{rgb}{0.000000,0.000000,0.000000}%
\pgfsetstrokecolor{textcolor}%
\pgfsetfillcolor{textcolor}%
\pgftext[x=0.329012in, y=1.800854in, left, base]{\color{textcolor}\rmfamily\fontsize{10.000000}{12.000000}\selectfont \(\displaystyle {20}\)}%
\end{pgfscope}%
\begin{pgfscope}%
\pgfsetbuttcap%
\pgfsetroundjoin%
\definecolor{currentfill}{rgb}{0.000000,0.000000,0.000000}%
\pgfsetfillcolor{currentfill}%
\pgfsetlinewidth{0.803000pt}%
\definecolor{currentstroke}{rgb}{0.000000,0.000000,0.000000}%
\pgfsetstrokecolor{currentstroke}%
\pgfsetdash{}{0pt}%
\pgfsys@defobject{currentmarker}{\pgfqpoint{-0.048611in}{0.000000in}}{\pgfqpoint{-0.000000in}{0.000000in}}{%
\pgfpathmoveto{\pgfqpoint{-0.000000in}{0.000000in}}%
\pgfpathlineto{\pgfqpoint{-0.048611in}{0.000000in}}%
\pgfusepath{stroke,fill}%
}%
\begin{pgfscope}%
\pgfsys@transformshift{0.565124in}{2.423767in}%
\pgfsys@useobject{currentmarker}{}%
\end{pgfscope}%
\end{pgfscope}%
\begin{pgfscope}%
\definecolor{textcolor}{rgb}{0.000000,0.000000,0.000000}%
\pgfsetstrokecolor{textcolor}%
\pgfsetfillcolor{textcolor}%
\pgftext[x=0.329012in, y=2.375541in, left, base]{\color{textcolor}\rmfamily\fontsize{10.000000}{12.000000}\selectfont \(\displaystyle {30}\)}%
\end{pgfscope}%
\begin{pgfscope}%
\pgfsetbuttcap%
\pgfsetroundjoin%
\definecolor{currentfill}{rgb}{0.000000,0.000000,0.000000}%
\pgfsetfillcolor{currentfill}%
\pgfsetlinewidth{0.803000pt}%
\definecolor{currentstroke}{rgb}{0.000000,0.000000,0.000000}%
\pgfsetstrokecolor{currentstroke}%
\pgfsetdash{}{0pt}%
\pgfsys@defobject{currentmarker}{\pgfqpoint{-0.048611in}{0.000000in}}{\pgfqpoint{-0.000000in}{0.000000in}}{%
\pgfpathmoveto{\pgfqpoint{-0.000000in}{0.000000in}}%
\pgfpathlineto{\pgfqpoint{-0.048611in}{0.000000in}}%
\pgfusepath{stroke,fill}%
}%
\begin{pgfscope}%
\pgfsys@transformshift{0.565124in}{2.998454in}%
\pgfsys@useobject{currentmarker}{}%
\end{pgfscope}%
\end{pgfscope}%
\begin{pgfscope}%
\definecolor{textcolor}{rgb}{0.000000,0.000000,0.000000}%
\pgfsetstrokecolor{textcolor}%
\pgfsetfillcolor{textcolor}%
\pgftext[x=0.329012in, y=2.950229in, left, base]{\color{textcolor}\rmfamily\fontsize{10.000000}{12.000000}\selectfont \(\displaystyle {40}\)}%
\end{pgfscope}%
\begin{pgfscope}%
\pgfsetbuttcap%
\pgfsetroundjoin%
\definecolor{currentfill}{rgb}{0.000000,0.000000,0.000000}%
\pgfsetfillcolor{currentfill}%
\pgfsetlinewidth{0.803000pt}%
\definecolor{currentstroke}{rgb}{0.000000,0.000000,0.000000}%
\pgfsetstrokecolor{currentstroke}%
\pgfsetdash{}{0pt}%
\pgfsys@defobject{currentmarker}{\pgfqpoint{-0.048611in}{0.000000in}}{\pgfqpoint{-0.000000in}{0.000000in}}{%
\pgfpathmoveto{\pgfqpoint{-0.000000in}{0.000000in}}%
\pgfpathlineto{\pgfqpoint{-0.048611in}{0.000000in}}%
\pgfusepath{stroke,fill}%
}%
\begin{pgfscope}%
\pgfsys@transformshift{0.565124in}{3.573141in}%
\pgfsys@useobject{currentmarker}{}%
\end{pgfscope}%
\end{pgfscope}%
\begin{pgfscope}%
\definecolor{textcolor}{rgb}{0.000000,0.000000,0.000000}%
\pgfsetstrokecolor{textcolor}%
\pgfsetfillcolor{textcolor}%
\pgftext[x=0.329012in, y=3.524916in, left, base]{\color{textcolor}\rmfamily\fontsize{10.000000}{12.000000}\selectfont \(\displaystyle {50}\)}%
\end{pgfscope}%
\begin{pgfscope}%
\definecolor{textcolor}{rgb}{0.000000,0.000000,0.000000}%
\pgfsetstrokecolor{textcolor}%
\pgfsetfillcolor{textcolor}%
\pgftext[x=0.273457in,y=2.199846in,,bottom,rotate=90.000000]{\color{textcolor}\rmfamily\fontsize{10.000000}{12.000000}\selectfont Regret}%
\end{pgfscope}%
\begin{pgfscope}%
\pgfpathrectangle{\pgfqpoint{0.565124in}{0.549691in}}{\pgfqpoint{2.474146in}{3.300309in}}%
\pgfusepath{clip}%
\pgfsetrectcap%
\pgfsetroundjoin%
\pgfsetlinewidth{1.505625pt}%
\definecolor{currentstroke}{rgb}{0.121569,0.466667,0.705882}%
\pgfsetstrokecolor{currentstroke}%
\pgfsetdash{}{0pt}%
\pgfpathmoveto{\pgfqpoint{0.677585in}{2.616402in}}%
\pgfpathlineto{\pgfqpoint{0.700077in}{2.759890in}}%
\pgfpathlineto{\pgfqpoint{0.722569in}{2.909078in}}%
\pgfpathlineto{\pgfqpoint{0.745062in}{3.072761in}}%
\pgfpathlineto{\pgfqpoint{0.767554in}{3.152079in}}%
\pgfpathlineto{\pgfqpoint{0.790046in}{3.380506in}}%
\pgfpathlineto{\pgfqpoint{0.812538in}{3.610197in}}%
\pgfpathlineto{\pgfqpoint{0.835031in}{3.641989in}}%
\pgfpathlineto{\pgfqpoint{0.857523in}{3.500845in}}%
\pgfpathlineto{\pgfqpoint{0.880015in}{2.794394in}}%
\pgfpathlineto{\pgfqpoint{0.902507in}{0.699705in}}%
\pgfpathlineto{\pgfqpoint{0.925000in}{2.801991in}}%
\pgfpathlineto{\pgfqpoint{0.947492in}{3.560820in}}%
\pgfpathlineto{\pgfqpoint{0.969984in}{3.699986in}}%
\pgfpathlineto{\pgfqpoint{0.992476in}{3.528086in}}%
\pgfpathlineto{\pgfqpoint{1.014969in}{3.347519in}}%
\pgfpathlineto{\pgfqpoint{1.037461in}{3.265385in}}%
\pgfpathlineto{\pgfqpoint{1.059953in}{3.114759in}}%
\pgfpathlineto{\pgfqpoint{1.082445in}{2.976754in}}%
\pgfpathlineto{\pgfqpoint{1.104938in}{2.631516in}}%
\pgfpathlineto{\pgfqpoint{1.127430in}{2.666055in}}%
\pgfpathlineto{\pgfqpoint{1.149922in}{2.575059in}}%
\pgfpathlineto{\pgfqpoint{1.172414in}{2.344322in}}%
\pgfpathlineto{\pgfqpoint{1.194906in}{2.265463in}}%
\pgfpathlineto{\pgfqpoint{1.217399in}{2.208259in}}%
\pgfpathlineto{\pgfqpoint{1.239891in}{2.124814in}}%
\pgfpathlineto{\pgfqpoint{1.262383in}{2.032612in}}%
\pgfpathlineto{\pgfqpoint{1.284875in}{1.996924in}}%
\pgfpathlineto{\pgfqpoint{1.307368in}{1.970821in}}%
\pgfpathlineto{\pgfqpoint{1.329860in}{1.911502in}}%
\pgfpathlineto{\pgfqpoint{1.352352in}{1.917812in}}%
\pgfpathlineto{\pgfqpoint{1.374844in}{1.877343in}}%
\pgfpathlineto{\pgfqpoint{1.397337in}{1.823173in}}%
\pgfpathlineto{\pgfqpoint{1.419829in}{1.743647in}}%
\pgfpathlineto{\pgfqpoint{1.442321in}{1.698282in}}%
\pgfpathlineto{\pgfqpoint{1.464813in}{1.739889in}}%
\pgfpathlineto{\pgfqpoint{1.487306in}{1.625503in}}%
\pgfpathlineto{\pgfqpoint{1.509798in}{1.562426in}}%
\pgfpathlineto{\pgfqpoint{1.532290in}{1.527439in}}%
\pgfpathlineto{\pgfqpoint{1.554782in}{1.601665in}}%
\pgfpathlineto{\pgfqpoint{1.577275in}{1.493463in}}%
\pgfpathlineto{\pgfqpoint{1.599767in}{1.531323in}}%
\pgfpathlineto{\pgfqpoint{1.622259in}{1.519163in}}%
\pgfpathlineto{\pgfqpoint{1.644751in}{1.500015in}}%
\pgfpathlineto{\pgfqpoint{1.667243in}{1.429696in}}%
\pgfpathlineto{\pgfqpoint{1.689736in}{1.450764in}}%
\pgfpathlineto{\pgfqpoint{1.712228in}{1.439534in}}%
\pgfpathlineto{\pgfqpoint{1.734720in}{1.401398in}}%
\pgfpathlineto{\pgfqpoint{1.757212in}{1.395904in}}%
\pgfpathlineto{\pgfqpoint{1.779705in}{1.365365in}}%
\pgfpathlineto{\pgfqpoint{1.802197in}{1.321287in}}%
\pgfpathlineto{\pgfqpoint{1.824689in}{1.339183in}}%
\pgfpathlineto{\pgfqpoint{1.847181in}{1.325815in}}%
\pgfpathlineto{\pgfqpoint{1.869674in}{1.304150in}}%
\pgfpathlineto{\pgfqpoint{1.892166in}{1.263082in}}%
\pgfpathlineto{\pgfqpoint{1.914658in}{1.317781in}}%
\pgfpathlineto{\pgfqpoint{1.937150in}{1.269128in}}%
\pgfpathlineto{\pgfqpoint{1.959643in}{1.311758in}}%
\pgfpathlineto{\pgfqpoint{1.982135in}{1.245336in}}%
\pgfpathlineto{\pgfqpoint{2.004627in}{1.254451in}}%
\pgfpathlineto{\pgfqpoint{2.027119in}{1.224395in}}%
\pgfpathlineto{\pgfqpoint{2.049612in}{1.227854in}}%
\pgfpathlineto{\pgfqpoint{2.072104in}{1.228050in}}%
\pgfpathlineto{\pgfqpoint{2.094596in}{1.202878in}}%
\pgfpathlineto{\pgfqpoint{2.117088in}{1.185063in}}%
\pgfpathlineto{\pgfqpoint{2.139581in}{1.170029in}}%
\pgfpathlineto{\pgfqpoint{2.162073in}{1.184373in}}%
\pgfpathlineto{\pgfqpoint{2.184565in}{1.170753in}}%
\pgfpathlineto{\pgfqpoint{2.207057in}{1.160351in}}%
\pgfpathlineto{\pgfqpoint{2.229549in}{1.146593in}}%
\pgfpathlineto{\pgfqpoint{2.252042in}{1.196235in}}%
\pgfpathlineto{\pgfqpoint{2.274534in}{1.158237in}}%
\pgfpathlineto{\pgfqpoint{2.297026in}{1.153639in}}%
\pgfpathlineto{\pgfqpoint{2.319518in}{1.132720in}}%
\pgfpathlineto{\pgfqpoint{2.342011in}{1.160926in}}%
\pgfpathlineto{\pgfqpoint{2.364503in}{1.123307in}}%
\pgfpathlineto{\pgfqpoint{2.386995in}{1.154857in}}%
\pgfpathlineto{\pgfqpoint{2.409487in}{1.116319in}}%
\pgfpathlineto{\pgfqpoint{2.431980in}{1.117066in}}%
\pgfpathlineto{\pgfqpoint{2.454472in}{1.102584in}}%
\pgfpathlineto{\pgfqpoint{2.476964in}{1.105204in}}%
\pgfpathlineto{\pgfqpoint{2.499456in}{1.096308in}}%
\pgfpathlineto{\pgfqpoint{2.521949in}{1.101067in}}%
\pgfpathlineto{\pgfqpoint{2.544441in}{1.063850in}}%
\pgfpathlineto{\pgfqpoint{2.566933in}{1.095205in}}%
\pgfpathlineto{\pgfqpoint{2.589425in}{1.092791in}}%
\pgfpathlineto{\pgfqpoint{2.611918in}{1.103273in}}%
\pgfpathlineto{\pgfqpoint{2.634410in}{1.097078in}}%
\pgfpathlineto{\pgfqpoint{2.656902in}{1.082516in}}%
\pgfpathlineto{\pgfqpoint{2.679394in}{1.068355in}}%
\pgfpathlineto{\pgfqpoint{2.701887in}{1.043598in}}%
\pgfpathlineto{\pgfqpoint{2.724379in}{1.029277in}}%
\pgfpathlineto{\pgfqpoint{2.746871in}{1.048425in}}%
\pgfpathlineto{\pgfqpoint{2.769363in}{1.072712in}}%
\pgfpathlineto{\pgfqpoint{2.791855in}{1.044380in}}%
\pgfpathlineto{\pgfqpoint{2.814348in}{1.048483in}}%
\pgfpathlineto{\pgfqpoint{2.836840in}{1.053575in}}%
\pgfpathlineto{\pgfqpoint{2.859332in}{1.048690in}}%
\pgfpathlineto{\pgfqpoint{2.881824in}{1.006174in}}%
\pgfpathlineto{\pgfqpoint{2.904317in}{0.988175in}}%
\pgfpathlineto{\pgfqpoint{2.926809in}{1.023484in}}%
\pgfusepath{stroke}%
\end{pgfscope}%
\begin{pgfscope}%
\pgfpathrectangle{\pgfqpoint{0.565124in}{0.549691in}}{\pgfqpoint{2.474146in}{3.300309in}}%
\pgfusepath{clip}%
\pgfsetrectcap%
\pgfsetroundjoin%
\pgfsetlinewidth{1.505625pt}%
\definecolor{currentstroke}{rgb}{1.000000,0.498039,0.054902}%
\pgfsetstrokecolor{currentstroke}%
\pgfsetdash{}{0pt}%
\pgfpathmoveto{\pgfqpoint{0.677585in}{1.130721in}}%
\pgfpathlineto{\pgfqpoint{0.700077in}{1.162202in}}%
\pgfpathlineto{\pgfqpoint{0.722569in}{1.279542in}}%
\pgfpathlineto{\pgfqpoint{0.745062in}{1.379962in}}%
\pgfpathlineto{\pgfqpoint{0.767554in}{1.577321in}}%
\pgfpathlineto{\pgfqpoint{0.790046in}{1.691673in}}%
\pgfpathlineto{\pgfqpoint{0.812538in}{1.759198in}}%
\pgfpathlineto{\pgfqpoint{0.835031in}{1.636974in}}%
\pgfpathlineto{\pgfqpoint{0.857523in}{1.614952in}}%
\pgfpathlineto{\pgfqpoint{0.880015in}{1.748348in}}%
\pgfpathlineto{\pgfqpoint{0.902507in}{0.699705in}}%
\pgfpathlineto{\pgfqpoint{0.925000in}{2.023037in}}%
\pgfpathlineto{\pgfqpoint{0.947492in}{1.984338in}}%
\pgfpathlineto{\pgfqpoint{0.969984in}{1.938650in}}%
\pgfpathlineto{\pgfqpoint{0.992476in}{2.031462in}}%
\pgfpathlineto{\pgfqpoint{1.014969in}{2.359402in}}%
\pgfpathlineto{\pgfqpoint{1.037461in}{2.001165in}}%
\pgfpathlineto{\pgfqpoint{1.059953in}{1.945409in}}%
\pgfpathlineto{\pgfqpoint{1.082445in}{1.952431in}}%
\pgfpathlineto{\pgfqpoint{1.104938in}{1.766210in}}%
\pgfpathlineto{\pgfqpoint{1.127430in}{1.721269in}}%
\pgfpathlineto{\pgfqpoint{1.149922in}{1.697879in}}%
\pgfpathlineto{\pgfqpoint{1.172414in}{1.782002in}}%
\pgfpathlineto{\pgfqpoint{1.194906in}{1.733981in}}%
\pgfpathlineto{\pgfqpoint{1.217399in}{1.695592in}}%
\pgfpathlineto{\pgfqpoint{1.239891in}{1.650353in}}%
\pgfpathlineto{\pgfqpoint{1.262383in}{1.536817in}}%
\pgfpathlineto{\pgfqpoint{1.284875in}{1.557874in}}%
\pgfpathlineto{\pgfqpoint{1.307368in}{1.625940in}}%
\pgfpathlineto{\pgfqpoint{1.329860in}{1.532611in}}%
\pgfpathlineto{\pgfqpoint{1.352352in}{1.511623in}}%
\pgfpathlineto{\pgfqpoint{1.374844in}{1.433224in}}%
\pgfpathlineto{\pgfqpoint{1.397337in}{1.559184in}}%
\pgfpathlineto{\pgfqpoint{1.419829in}{1.479291in}}%
\pgfpathlineto{\pgfqpoint{1.442321in}{1.382158in}}%
\pgfpathlineto{\pgfqpoint{1.464813in}{1.447373in}}%
\pgfpathlineto{\pgfqpoint{1.487306in}{1.387031in}}%
\pgfpathlineto{\pgfqpoint{1.509798in}{1.397329in}}%
\pgfpathlineto{\pgfqpoint{1.532290in}{1.325011in}}%
\pgfpathlineto{\pgfqpoint{1.554782in}{1.361676in}}%
\pgfpathlineto{\pgfqpoint{1.577275in}{1.260370in}}%
\pgfpathlineto{\pgfqpoint{1.599767in}{1.306851in}}%
\pgfpathlineto{\pgfqpoint{1.622259in}{1.280829in}}%
\pgfpathlineto{\pgfqpoint{1.644751in}{1.293300in}}%
\pgfpathlineto{\pgfqpoint{1.667243in}{1.284323in}}%
\pgfpathlineto{\pgfqpoint{1.689736in}{1.301920in}}%
\pgfpathlineto{\pgfqpoint{1.712228in}{1.255818in}}%
\pgfpathlineto{\pgfqpoint{1.734720in}{1.324425in}}%
\pgfpathlineto{\pgfqpoint{1.757212in}{1.275358in}}%
\pgfpathlineto{\pgfqpoint{1.779705in}{1.272576in}}%
\pgfpathlineto{\pgfqpoint{1.802197in}{1.281289in}}%
\pgfpathlineto{\pgfqpoint{1.824689in}{1.248704in}}%
\pgfpathlineto{\pgfqpoint{1.847181in}{1.246646in}}%
\pgfpathlineto{\pgfqpoint{1.869674in}{1.227050in}}%
\pgfpathlineto{\pgfqpoint{1.892166in}{1.184189in}}%
\pgfpathlineto{\pgfqpoint{1.914658in}{1.220544in}}%
\pgfpathlineto{\pgfqpoint{1.937150in}{1.218372in}}%
\pgfpathlineto{\pgfqpoint{1.959643in}{1.171845in}}%
\pgfpathlineto{\pgfqpoint{1.982135in}{1.200097in}}%
\pgfpathlineto{\pgfqpoint{2.004627in}{1.184615in}}%
\pgfpathlineto{\pgfqpoint{2.027119in}{1.201982in}}%
\pgfpathlineto{\pgfqpoint{2.049612in}{1.170408in}}%
\pgfpathlineto{\pgfqpoint{2.072104in}{1.145570in}}%
\pgfpathlineto{\pgfqpoint{2.094596in}{1.108457in}}%
\pgfpathlineto{\pgfqpoint{2.117088in}{1.150927in}}%
\pgfpathlineto{\pgfqpoint{2.139581in}{1.142846in}}%
\pgfpathlineto{\pgfqpoint{2.162073in}{1.119365in}}%
\pgfpathlineto{\pgfqpoint{2.184565in}{1.105894in}}%
\pgfpathlineto{\pgfqpoint{2.207057in}{1.097688in}}%
\pgfpathlineto{\pgfqpoint{2.229549in}{1.120824in}}%
\pgfpathlineto{\pgfqpoint{2.252042in}{1.088653in}}%
\pgfpathlineto{\pgfqpoint{2.274534in}{1.098641in}}%
\pgfpathlineto{\pgfqpoint{2.297026in}{1.123709in}}%
\pgfpathlineto{\pgfqpoint{2.319518in}{1.059586in}}%
\pgfpathlineto{\pgfqpoint{2.342011in}{1.085895in}}%
\pgfpathlineto{\pgfqpoint{2.364503in}{1.077734in}}%
\pgfpathlineto{\pgfqpoint{2.386995in}{1.042587in}}%
\pgfpathlineto{\pgfqpoint{2.409487in}{1.054712in}}%
\pgfpathlineto{\pgfqpoint{2.431980in}{1.027185in}}%
\pgfpathlineto{\pgfqpoint{2.454472in}{0.998692in}}%
\pgfpathlineto{\pgfqpoint{2.476964in}{1.039230in}}%
\pgfpathlineto{\pgfqpoint{2.499456in}{1.005726in}}%
\pgfpathlineto{\pgfqpoint{2.521949in}{0.972796in}}%
\pgfpathlineto{\pgfqpoint{2.544441in}{0.974911in}}%
\pgfpathlineto{\pgfqpoint{2.566933in}{0.976980in}}%
\pgfpathlineto{\pgfqpoint{2.589425in}{0.958314in}}%
\pgfpathlineto{\pgfqpoint{2.611918in}{0.966130in}}%
\pgfpathlineto{\pgfqpoint{2.634410in}{0.974946in}}%
\pgfpathlineto{\pgfqpoint{2.656902in}{0.955211in}}%
\pgfpathlineto{\pgfqpoint{2.679394in}{0.940327in}}%
\pgfpathlineto{\pgfqpoint{2.701887in}{0.923144in}}%
\pgfpathlineto{\pgfqpoint{2.724379in}{0.915696in}}%
\pgfpathlineto{\pgfqpoint{2.746871in}{0.896685in}}%
\pgfpathlineto{\pgfqpoint{2.769363in}{0.898133in}}%
\pgfpathlineto{\pgfqpoint{2.791855in}{0.872525in}}%
\pgfpathlineto{\pgfqpoint{2.814348in}{0.821826in}}%
\pgfpathlineto{\pgfqpoint{2.836840in}{0.791632in}}%
\pgfpathlineto{\pgfqpoint{2.859332in}{0.779702in}}%
\pgfpathlineto{\pgfqpoint{2.881824in}{0.771518in}}%
\pgfpathlineto{\pgfqpoint{2.904317in}{0.765174in}}%
\pgfpathlineto{\pgfqpoint{2.926809in}{0.761771in}}%
\pgfusepath{stroke}%
\end{pgfscope}%
\begin{pgfscope}%
\pgfsetrectcap%
\pgfsetmiterjoin%
\pgfsetlinewidth{0.803000pt}%
\definecolor{currentstroke}{rgb}{0.000000,0.000000,0.000000}%
\pgfsetstrokecolor{currentstroke}%
\pgfsetdash{}{0pt}%
\pgfpathmoveto{\pgfqpoint{0.565124in}{0.549691in}}%
\pgfpathlineto{\pgfqpoint{0.565124in}{3.850000in}}%
\pgfusepath{stroke}%
\end{pgfscope}%
\begin{pgfscope}%
\pgfsetrectcap%
\pgfsetmiterjoin%
\pgfsetlinewidth{0.803000pt}%
\definecolor{currentstroke}{rgb}{0.000000,0.000000,0.000000}%
\pgfsetstrokecolor{currentstroke}%
\pgfsetdash{}{0pt}%
\pgfpathmoveto{\pgfqpoint{3.039270in}{0.549691in}}%
\pgfpathlineto{\pgfqpoint{3.039270in}{3.850000in}}%
\pgfusepath{stroke}%
\end{pgfscope}%
\begin{pgfscope}%
\pgfsetrectcap%
\pgfsetmiterjoin%
\pgfsetlinewidth{0.803000pt}%
\definecolor{currentstroke}{rgb}{0.000000,0.000000,0.000000}%
\pgfsetstrokecolor{currentstroke}%
\pgfsetdash{}{0pt}%
\pgfpathmoveto{\pgfqpoint{0.565124in}{0.549691in}}%
\pgfpathlineto{\pgfqpoint{3.039270in}{0.549691in}}%
\pgfusepath{stroke}%
\end{pgfscope}%
\begin{pgfscope}%
\pgfsetrectcap%
\pgfsetmiterjoin%
\pgfsetlinewidth{0.803000pt}%
\definecolor{currentstroke}{rgb}{0.000000,0.000000,0.000000}%
\pgfsetstrokecolor{currentstroke}%
\pgfsetdash{}{0pt}%
\pgfpathmoveto{\pgfqpoint{0.565124in}{3.850000in}}%
\pgfpathlineto{\pgfqpoint{3.039270in}{3.850000in}}%
\pgfusepath{stroke}%
\end{pgfscope}%
\end{pgfpicture}%
\makeatother%
\endgroup%

		\end{center}
		\caption{$\mu_1 = 0.1$}
	\end{subfigure}
	\hfill
	\begin{subfigure}[h]{0.49\textwidth}
		\begin{center}
			%% Creator: Matplotlib, PGF backend
%%
%% To include the figure in your LaTeX document, write
%%   \input{<filename>.pgf}
%%
%% Make sure the required packages are loaded in your preamble
%%   \usepackage{pgf}
%%
%% and, on pdftex
%%   \usepackage[utf8]{inputenc}\DeclareUnicodeCharacter{2212}{-}
%%
%% or, on luatex and xetex
%%   \usepackage{unicode-math}
%%
%% Figures using additional raster images can only be included by \input if
%% they are in the same directory as the main LaTeX file. For loading figures
%% from other directories you can use the `import` package
%%   \usepackage{import}
%%
%% and then include the figures with
%%   \import{<path to file>}{<filename>.pgf}
%%
%% Matplotlib used the following preamble
%%
\begingroup%
\makeatletter%
\begin{pgfpicture}%
\pgfpathrectangle{\pgfpointorigin}{\pgfqpoint{3.200000in}{4.000000in}}%
\pgfusepath{use as bounding box, clip}%
\begin{pgfscope}%
\pgfsetbuttcap%
\pgfsetmiterjoin%
\pgfsetlinewidth{0.000000pt}%
\definecolor{currentstroke}{rgb}{1.000000,1.000000,1.000000}%
\pgfsetstrokecolor{currentstroke}%
\pgfsetstrokeopacity{0.000000}%
\pgfsetdash{}{0pt}%
\pgfpathmoveto{\pgfqpoint{0.000000in}{0.000000in}}%
\pgfpathlineto{\pgfqpoint{3.200000in}{0.000000in}}%
\pgfpathlineto{\pgfqpoint{3.200000in}{4.000000in}}%
\pgfpathlineto{\pgfqpoint{0.000000in}{4.000000in}}%
\pgfpathclose%
\pgfusepath{}%
\end{pgfscope}%
\begin{pgfscope}%
\pgfsetbuttcap%
\pgfsetmiterjoin%
\definecolor{currentfill}{rgb}{1.000000,1.000000,1.000000}%
\pgfsetfillcolor{currentfill}%
\pgfsetlinewidth{0.000000pt}%
\definecolor{currentstroke}{rgb}{0.000000,0.000000,0.000000}%
\pgfsetstrokecolor{currentstroke}%
\pgfsetstrokeopacity{0.000000}%
\pgfsetdash{}{0pt}%
\pgfpathmoveto{\pgfqpoint{0.565124in}{0.549691in}}%
\pgfpathlineto{\pgfqpoint{3.039270in}{0.549691in}}%
\pgfpathlineto{\pgfqpoint{3.039270in}{3.850000in}}%
\pgfpathlineto{\pgfqpoint{0.565124in}{3.850000in}}%
\pgfpathclose%
\pgfusepath{fill}%
\end{pgfscope}%
\begin{pgfscope}%
\pgfsetbuttcap%
\pgfsetroundjoin%
\definecolor{currentfill}{rgb}{0.000000,0.000000,0.000000}%
\pgfsetfillcolor{currentfill}%
\pgfsetlinewidth{0.803000pt}%
\definecolor{currentstroke}{rgb}{0.000000,0.000000,0.000000}%
\pgfsetstrokecolor{currentstroke}%
\pgfsetdash{}{0pt}%
\pgfsys@defobject{currentmarker}{\pgfqpoint{0.000000in}{-0.048611in}}{\pgfqpoint{0.000000in}{0.000000in}}{%
\pgfpathmoveto{\pgfqpoint{0.000000in}{0.000000in}}%
\pgfpathlineto{\pgfqpoint{0.000000in}{-0.048611in}}%
\pgfusepath{stroke,fill}%
}%
\begin{pgfscope}%
\pgfsys@transformshift{0.677585in}{0.549691in}%
\pgfsys@useobject{currentmarker}{}%
\end{pgfscope}%
\end{pgfscope}%
\begin{pgfscope}%
\definecolor{textcolor}{rgb}{0.000000,0.000000,0.000000}%
\pgfsetstrokecolor{textcolor}%
\pgfsetfillcolor{textcolor}%
\pgftext[x=0.677585in,y=0.452469in,,top]{\color{textcolor}\rmfamily\fontsize{10.000000}{12.000000}\selectfont \(\displaystyle {−0.50}\)}%
\end{pgfscope}%
\begin{pgfscope}%
\pgfsetbuttcap%
\pgfsetroundjoin%
\definecolor{currentfill}{rgb}{0.000000,0.000000,0.000000}%
\pgfsetfillcolor{currentfill}%
\pgfsetlinewidth{0.803000pt}%
\definecolor{currentstroke}{rgb}{0.000000,0.000000,0.000000}%
\pgfsetstrokecolor{currentstroke}%
\pgfsetdash{}{0pt}%
\pgfsys@defobject{currentmarker}{\pgfqpoint{0.000000in}{-0.048611in}}{\pgfqpoint{0.000000in}{0.000000in}}{%
\pgfpathmoveto{\pgfqpoint{0.000000in}{0.000000in}}%
\pgfpathlineto{\pgfqpoint{0.000000in}{-0.048611in}}%
\pgfusepath{stroke,fill}%
}%
\begin{pgfscope}%
\pgfsys@transformshift{1.239891in}{0.549691in}%
\pgfsys@useobject{currentmarker}{}%
\end{pgfscope}%
\end{pgfscope}%
\begin{pgfscope}%
\definecolor{textcolor}{rgb}{0.000000,0.000000,0.000000}%
\pgfsetstrokecolor{textcolor}%
\pgfsetfillcolor{textcolor}%
\pgftext[x=1.239891in,y=0.452469in,,top]{\color{textcolor}\rmfamily\fontsize{10.000000}{12.000000}\selectfont \(\displaystyle {−0.25}\)}%
\end{pgfscope}%
\begin{pgfscope}%
\pgfsetbuttcap%
\pgfsetroundjoin%
\definecolor{currentfill}{rgb}{0.000000,0.000000,0.000000}%
\pgfsetfillcolor{currentfill}%
\pgfsetlinewidth{0.803000pt}%
\definecolor{currentstroke}{rgb}{0.000000,0.000000,0.000000}%
\pgfsetstrokecolor{currentstroke}%
\pgfsetdash{}{0pt}%
\pgfsys@defobject{currentmarker}{\pgfqpoint{0.000000in}{-0.048611in}}{\pgfqpoint{0.000000in}{0.000000in}}{%
\pgfpathmoveto{\pgfqpoint{0.000000in}{0.000000in}}%
\pgfpathlineto{\pgfqpoint{0.000000in}{-0.048611in}}%
\pgfusepath{stroke,fill}%
}%
\begin{pgfscope}%
\pgfsys@transformshift{1.802197in}{0.549691in}%
\pgfsys@useobject{currentmarker}{}%
\end{pgfscope}%
\end{pgfscope}%
\begin{pgfscope}%
\definecolor{textcolor}{rgb}{0.000000,0.000000,0.000000}%
\pgfsetstrokecolor{textcolor}%
\pgfsetfillcolor{textcolor}%
\pgftext[x=1.802197in,y=0.452469in,,top]{\color{textcolor}\rmfamily\fontsize{10.000000}{12.000000}\selectfont \(\displaystyle {0.00}\)}%
\end{pgfscope}%
\begin{pgfscope}%
\pgfsetbuttcap%
\pgfsetroundjoin%
\definecolor{currentfill}{rgb}{0.000000,0.000000,0.000000}%
\pgfsetfillcolor{currentfill}%
\pgfsetlinewidth{0.803000pt}%
\definecolor{currentstroke}{rgb}{0.000000,0.000000,0.000000}%
\pgfsetstrokecolor{currentstroke}%
\pgfsetdash{}{0pt}%
\pgfsys@defobject{currentmarker}{\pgfqpoint{0.000000in}{-0.048611in}}{\pgfqpoint{0.000000in}{0.000000in}}{%
\pgfpathmoveto{\pgfqpoint{0.000000in}{0.000000in}}%
\pgfpathlineto{\pgfqpoint{0.000000in}{-0.048611in}}%
\pgfusepath{stroke,fill}%
}%
\begin{pgfscope}%
\pgfsys@transformshift{2.364503in}{0.549691in}%
\pgfsys@useobject{currentmarker}{}%
\end{pgfscope}%
\end{pgfscope}%
\begin{pgfscope}%
\definecolor{textcolor}{rgb}{0.000000,0.000000,0.000000}%
\pgfsetstrokecolor{textcolor}%
\pgfsetfillcolor{textcolor}%
\pgftext[x=2.364503in,y=0.452469in,,top]{\color{textcolor}\rmfamily\fontsize{10.000000}{12.000000}\selectfont \(\displaystyle {0.25}\)}%
\end{pgfscope}%
\begin{pgfscope}%
\pgfsetbuttcap%
\pgfsetroundjoin%
\definecolor{currentfill}{rgb}{0.000000,0.000000,0.000000}%
\pgfsetfillcolor{currentfill}%
\pgfsetlinewidth{0.803000pt}%
\definecolor{currentstroke}{rgb}{0.000000,0.000000,0.000000}%
\pgfsetstrokecolor{currentstroke}%
\pgfsetdash{}{0pt}%
\pgfsys@defobject{currentmarker}{\pgfqpoint{0.000000in}{-0.048611in}}{\pgfqpoint{0.000000in}{0.000000in}}{%
\pgfpathmoveto{\pgfqpoint{0.000000in}{0.000000in}}%
\pgfpathlineto{\pgfqpoint{0.000000in}{-0.048611in}}%
\pgfusepath{stroke,fill}%
}%
\begin{pgfscope}%
\pgfsys@transformshift{2.926809in}{0.549691in}%
\pgfsys@useobject{currentmarker}{}%
\end{pgfscope}%
\end{pgfscope}%
\begin{pgfscope}%
\definecolor{textcolor}{rgb}{0.000000,0.000000,0.000000}%
\pgfsetstrokecolor{textcolor}%
\pgfsetfillcolor{textcolor}%
\pgftext[x=2.926809in,y=0.452469in,,top]{\color{textcolor}\rmfamily\fontsize{10.000000}{12.000000}\selectfont \(\displaystyle {0.50}\)}%
\end{pgfscope}%
\begin{pgfscope}%
\definecolor{textcolor}{rgb}{0.000000,0.000000,0.000000}%
\pgfsetstrokecolor{textcolor}%
\pgfsetfillcolor{textcolor}%
\pgftext[x=1.802197in,y=0.273457in,,top]{\color{textcolor}\rmfamily\fontsize{10.000000}{12.000000}\selectfont \(\displaystyle \Delta\)}%
\end{pgfscope}%
\begin{pgfscope}%
\pgfsetbuttcap%
\pgfsetroundjoin%
\definecolor{currentfill}{rgb}{0.000000,0.000000,0.000000}%
\pgfsetfillcolor{currentfill}%
\pgfsetlinewidth{0.803000pt}%
\definecolor{currentstroke}{rgb}{0.000000,0.000000,0.000000}%
\pgfsetstrokecolor{currentstroke}%
\pgfsetdash{}{0pt}%
\pgfsys@defobject{currentmarker}{\pgfqpoint{-0.048611in}{0.000000in}}{\pgfqpoint{-0.000000in}{0.000000in}}{%
\pgfpathmoveto{\pgfqpoint{-0.000000in}{0.000000in}}%
\pgfpathlineto{\pgfqpoint{-0.048611in}{0.000000in}}%
\pgfusepath{stroke,fill}%
}%
\begin{pgfscope}%
\pgfsys@transformshift{0.565124in}{0.699705in}%
\pgfsys@useobject{currentmarker}{}%
\end{pgfscope}%
\end{pgfscope}%
\begin{pgfscope}%
\definecolor{textcolor}{rgb}{0.000000,0.000000,0.000000}%
\pgfsetstrokecolor{textcolor}%
\pgfsetfillcolor{textcolor}%
\pgftext[x=0.398457in, y=0.651480in, left, base]{\color{textcolor}\rmfamily\fontsize{10.000000}{12.000000}\selectfont \(\displaystyle {0}\)}%
\end{pgfscope}%
\begin{pgfscope}%
\pgfsetbuttcap%
\pgfsetroundjoin%
\definecolor{currentfill}{rgb}{0.000000,0.000000,0.000000}%
\pgfsetfillcolor{currentfill}%
\pgfsetlinewidth{0.803000pt}%
\definecolor{currentstroke}{rgb}{0.000000,0.000000,0.000000}%
\pgfsetstrokecolor{currentstroke}%
\pgfsetdash{}{0pt}%
\pgfsys@defobject{currentmarker}{\pgfqpoint{-0.048611in}{0.000000in}}{\pgfqpoint{-0.000000in}{0.000000in}}{%
\pgfpathmoveto{\pgfqpoint{-0.000000in}{0.000000in}}%
\pgfpathlineto{\pgfqpoint{-0.048611in}{0.000000in}}%
\pgfusepath{stroke,fill}%
}%
\begin{pgfscope}%
\pgfsys@transformshift{0.565124in}{1.274551in}%
\pgfsys@useobject{currentmarker}{}%
\end{pgfscope}%
\end{pgfscope}%
\begin{pgfscope}%
\definecolor{textcolor}{rgb}{0.000000,0.000000,0.000000}%
\pgfsetstrokecolor{textcolor}%
\pgfsetfillcolor{textcolor}%
\pgftext[x=0.329012in, y=1.226326in, left, base]{\color{textcolor}\rmfamily\fontsize{10.000000}{12.000000}\selectfont \(\displaystyle {10}\)}%
\end{pgfscope}%
\begin{pgfscope}%
\pgfsetbuttcap%
\pgfsetroundjoin%
\definecolor{currentfill}{rgb}{0.000000,0.000000,0.000000}%
\pgfsetfillcolor{currentfill}%
\pgfsetlinewidth{0.803000pt}%
\definecolor{currentstroke}{rgb}{0.000000,0.000000,0.000000}%
\pgfsetstrokecolor{currentstroke}%
\pgfsetdash{}{0pt}%
\pgfsys@defobject{currentmarker}{\pgfqpoint{-0.048611in}{0.000000in}}{\pgfqpoint{-0.000000in}{0.000000in}}{%
\pgfpathmoveto{\pgfqpoint{-0.000000in}{0.000000in}}%
\pgfpathlineto{\pgfqpoint{-0.048611in}{0.000000in}}%
\pgfusepath{stroke,fill}%
}%
\begin{pgfscope}%
\pgfsys@transformshift{0.565124in}{1.849397in}%
\pgfsys@useobject{currentmarker}{}%
\end{pgfscope}%
\end{pgfscope}%
\begin{pgfscope}%
\definecolor{textcolor}{rgb}{0.000000,0.000000,0.000000}%
\pgfsetstrokecolor{textcolor}%
\pgfsetfillcolor{textcolor}%
\pgftext[x=0.329012in, y=1.801171in, left, base]{\color{textcolor}\rmfamily\fontsize{10.000000}{12.000000}\selectfont \(\displaystyle {20}\)}%
\end{pgfscope}%
\begin{pgfscope}%
\pgfsetbuttcap%
\pgfsetroundjoin%
\definecolor{currentfill}{rgb}{0.000000,0.000000,0.000000}%
\pgfsetfillcolor{currentfill}%
\pgfsetlinewidth{0.803000pt}%
\definecolor{currentstroke}{rgb}{0.000000,0.000000,0.000000}%
\pgfsetstrokecolor{currentstroke}%
\pgfsetdash{}{0pt}%
\pgfsys@defobject{currentmarker}{\pgfqpoint{-0.048611in}{0.000000in}}{\pgfqpoint{-0.000000in}{0.000000in}}{%
\pgfpathmoveto{\pgfqpoint{-0.000000in}{0.000000in}}%
\pgfpathlineto{\pgfqpoint{-0.048611in}{0.000000in}}%
\pgfusepath{stroke,fill}%
}%
\begin{pgfscope}%
\pgfsys@transformshift{0.565124in}{2.424242in}%
\pgfsys@useobject{currentmarker}{}%
\end{pgfscope}%
\end{pgfscope}%
\begin{pgfscope}%
\definecolor{textcolor}{rgb}{0.000000,0.000000,0.000000}%
\pgfsetstrokecolor{textcolor}%
\pgfsetfillcolor{textcolor}%
\pgftext[x=0.329012in, y=2.376017in, left, base]{\color{textcolor}\rmfamily\fontsize{10.000000}{12.000000}\selectfont \(\displaystyle {30}\)}%
\end{pgfscope}%
\begin{pgfscope}%
\pgfsetbuttcap%
\pgfsetroundjoin%
\definecolor{currentfill}{rgb}{0.000000,0.000000,0.000000}%
\pgfsetfillcolor{currentfill}%
\pgfsetlinewidth{0.803000pt}%
\definecolor{currentstroke}{rgb}{0.000000,0.000000,0.000000}%
\pgfsetstrokecolor{currentstroke}%
\pgfsetdash{}{0pt}%
\pgfsys@defobject{currentmarker}{\pgfqpoint{-0.048611in}{0.000000in}}{\pgfqpoint{-0.000000in}{0.000000in}}{%
\pgfpathmoveto{\pgfqpoint{-0.000000in}{0.000000in}}%
\pgfpathlineto{\pgfqpoint{-0.048611in}{0.000000in}}%
\pgfusepath{stroke,fill}%
}%
\begin{pgfscope}%
\pgfsys@transformshift{0.565124in}{2.999088in}%
\pgfsys@useobject{currentmarker}{}%
\end{pgfscope}%
\end{pgfscope}%
\begin{pgfscope}%
\definecolor{textcolor}{rgb}{0.000000,0.000000,0.000000}%
\pgfsetstrokecolor{textcolor}%
\pgfsetfillcolor{textcolor}%
\pgftext[x=0.329012in, y=2.950863in, left, base]{\color{textcolor}\rmfamily\fontsize{10.000000}{12.000000}\selectfont \(\displaystyle {40}\)}%
\end{pgfscope}%
\begin{pgfscope}%
\pgfsetbuttcap%
\pgfsetroundjoin%
\definecolor{currentfill}{rgb}{0.000000,0.000000,0.000000}%
\pgfsetfillcolor{currentfill}%
\pgfsetlinewidth{0.803000pt}%
\definecolor{currentstroke}{rgb}{0.000000,0.000000,0.000000}%
\pgfsetstrokecolor{currentstroke}%
\pgfsetdash{}{0pt}%
\pgfsys@defobject{currentmarker}{\pgfqpoint{-0.048611in}{0.000000in}}{\pgfqpoint{-0.000000in}{0.000000in}}{%
\pgfpathmoveto{\pgfqpoint{-0.000000in}{0.000000in}}%
\pgfpathlineto{\pgfqpoint{-0.048611in}{0.000000in}}%
\pgfusepath{stroke,fill}%
}%
\begin{pgfscope}%
\pgfsys@transformshift{0.565124in}{3.573934in}%
\pgfsys@useobject{currentmarker}{}%
\end{pgfscope}%
\end{pgfscope}%
\begin{pgfscope}%
\definecolor{textcolor}{rgb}{0.000000,0.000000,0.000000}%
\pgfsetstrokecolor{textcolor}%
\pgfsetfillcolor{textcolor}%
\pgftext[x=0.329012in, y=3.525709in, left, base]{\color{textcolor}\rmfamily\fontsize{10.000000}{12.000000}\selectfont \(\displaystyle {50}\)}%
\end{pgfscope}%
\begin{pgfscope}%
\definecolor{textcolor}{rgb}{0.000000,0.000000,0.000000}%
\pgfsetstrokecolor{textcolor}%
\pgfsetfillcolor{textcolor}%
\pgftext[x=0.273457in,y=2.199846in,,bottom,rotate=90.000000]{\color{textcolor}\rmfamily\fontsize{10.000000}{12.000000}\selectfont Regret}%
\end{pgfscope}%
\begin{pgfscope}%
\pgfpathrectangle{\pgfqpoint{0.565124in}{0.549691in}}{\pgfqpoint{2.474146in}{3.300309in}}%
\pgfusepath{clip}%
\pgfsetrectcap%
\pgfsetroundjoin%
\pgfsetlinewidth{1.505625pt}%
\definecolor{currentstroke}{rgb}{0.121569,0.466667,0.705882}%
\pgfsetstrokecolor{currentstroke}%
\pgfsetdash{}{0pt}%
\pgfpathmoveto{\pgfqpoint{0.677585in}{1.011157in}}%
\pgfpathlineto{\pgfqpoint{0.700077in}{1.025091in}}%
\pgfpathlineto{\pgfqpoint{0.722569in}{1.016376in}}%
\pgfpathlineto{\pgfqpoint{0.745062in}{1.009777in}}%
\pgfpathlineto{\pgfqpoint{0.767554in}{1.021044in}}%
\pgfpathlineto{\pgfqpoint{0.790046in}{1.025125in}}%
\pgfpathlineto{\pgfqpoint{0.812538in}{1.038680in}}%
\pgfpathlineto{\pgfqpoint{0.835031in}{1.041324in}}%
\pgfpathlineto{\pgfqpoint{0.857523in}{1.057006in}}%
\pgfpathlineto{\pgfqpoint{0.880015in}{1.043336in}}%
\pgfpathlineto{\pgfqpoint{0.902507in}{1.078643in}}%
\pgfpathlineto{\pgfqpoint{0.925000in}{1.080265in}}%
\pgfpathlineto{\pgfqpoint{0.947492in}{1.066480in}}%
\pgfpathlineto{\pgfqpoint{0.969984in}{1.060007in}}%
\pgfpathlineto{\pgfqpoint{0.992476in}{1.089405in}}%
\pgfpathlineto{\pgfqpoint{1.014969in}{1.047199in}}%
\pgfpathlineto{\pgfqpoint{1.037461in}{1.100419in}}%
\pgfpathlineto{\pgfqpoint{1.059953in}{1.050522in}}%
\pgfpathlineto{\pgfqpoint{1.082445in}{1.100350in}}%
\pgfpathlineto{\pgfqpoint{1.104938in}{1.049890in}}%
\pgfpathlineto{\pgfqpoint{1.127430in}{1.093245in}}%
\pgfpathlineto{\pgfqpoint{1.149922in}{1.121734in}}%
\pgfpathlineto{\pgfqpoint{1.172414in}{1.099200in}}%
\pgfpathlineto{\pgfqpoint{1.194906in}{1.123366in}}%
\pgfpathlineto{\pgfqpoint{1.217399in}{1.110973in}}%
\pgfpathlineto{\pgfqpoint{1.239891in}{1.142854in}}%
\pgfpathlineto{\pgfqpoint{1.262383in}{1.160318in}}%
\pgfpathlineto{\pgfqpoint{1.284875in}{1.141532in}}%
\pgfpathlineto{\pgfqpoint{1.307368in}{1.108145in}}%
\pgfpathlineto{\pgfqpoint{1.329860in}{1.178701in}}%
\pgfpathlineto{\pgfqpoint{1.352352in}{1.165330in}}%
\pgfpathlineto{\pgfqpoint{1.374844in}{1.181989in}}%
\pgfpathlineto{\pgfqpoint{1.397337in}{1.208490in}}%
\pgfpathlineto{\pgfqpoint{1.419829in}{1.143360in}}%
\pgfpathlineto{\pgfqpoint{1.442321in}{1.173562in}}%
\pgfpathlineto{\pgfqpoint{1.464813in}{1.206202in}}%
\pgfpathlineto{\pgfqpoint{1.487306in}{1.199476in}}%
\pgfpathlineto{\pgfqpoint{1.509798in}{1.206064in}}%
\pgfpathlineto{\pgfqpoint{1.532290in}{1.189934in}}%
\pgfpathlineto{\pgfqpoint{1.554782in}{1.233863in}}%
\pgfpathlineto{\pgfqpoint{1.577275in}{1.232012in}}%
\pgfpathlineto{\pgfqpoint{1.599767in}{1.229253in}}%
\pgfpathlineto{\pgfqpoint{1.622259in}{1.225620in}}%
\pgfpathlineto{\pgfqpoint{1.644751in}{1.258432in}}%
\pgfpathlineto{\pgfqpoint{1.667243in}{1.275632in}}%
\pgfpathlineto{\pgfqpoint{1.689736in}{1.290014in}}%
\pgfpathlineto{\pgfqpoint{1.712228in}{1.338106in}}%
\pgfpathlineto{\pgfqpoint{1.734720in}{1.296901in}}%
\pgfpathlineto{\pgfqpoint{1.757212in}{1.296050in}}%
\pgfpathlineto{\pgfqpoint{1.779705in}{1.305420in}}%
\pgfpathlineto{\pgfqpoint{1.802197in}{1.374804in}}%
\pgfpathlineto{\pgfqpoint{1.824689in}{1.342233in}}%
\pgfpathlineto{\pgfqpoint{1.847181in}{1.394786in}}%
\pgfpathlineto{\pgfqpoint{1.869674in}{1.390106in}}%
\pgfpathlineto{\pgfqpoint{1.892166in}{1.395866in}}%
\pgfpathlineto{\pgfqpoint{1.914658in}{1.541509in}}%
\pgfpathlineto{\pgfqpoint{1.937150in}{1.553420in}}%
\pgfpathlineto{\pgfqpoint{1.959643in}{1.519585in}}%
\pgfpathlineto{\pgfqpoint{1.982135in}{1.496579in}}%
\pgfpathlineto{\pgfqpoint{2.004627in}{1.555432in}}%
\pgfpathlineto{\pgfqpoint{2.027119in}{1.484370in}}%
\pgfpathlineto{\pgfqpoint{2.049612in}{1.513227in}}%
\pgfpathlineto{\pgfqpoint{2.072104in}{1.506099in}}%
\pgfpathlineto{\pgfqpoint{2.094596in}{1.512066in}}%
\pgfpathlineto{\pgfqpoint{2.117088in}{1.793752in}}%
\pgfpathlineto{\pgfqpoint{2.139581in}{1.657973in}}%
\pgfpathlineto{\pgfqpoint{2.162073in}{1.639785in}}%
\pgfpathlineto{\pgfqpoint{2.184565in}{1.727805in}}%
\pgfpathlineto{\pgfqpoint{2.207057in}{1.791613in}}%
\pgfpathlineto{\pgfqpoint{2.229549in}{1.794372in}}%
\pgfpathlineto{\pgfqpoint{2.252042in}{1.851696in}}%
\pgfpathlineto{\pgfqpoint{2.274534in}{1.832979in}}%
\pgfpathlineto{\pgfqpoint{2.297026in}{1.901018in}}%
\pgfpathlineto{\pgfqpoint{2.319518in}{2.103605in}}%
\pgfpathlineto{\pgfqpoint{2.342011in}{2.020655in}}%
\pgfpathlineto{\pgfqpoint{2.364503in}{2.053639in}}%
\pgfpathlineto{\pgfqpoint{2.386995in}{2.171655in}}%
\pgfpathlineto{\pgfqpoint{2.409487in}{2.335245in}}%
\pgfpathlineto{\pgfqpoint{2.431980in}{2.327530in}}%
\pgfpathlineto{\pgfqpoint{2.454472in}{2.343258in}}%
\pgfpathlineto{\pgfqpoint{2.476964in}{2.605779in}}%
\pgfpathlineto{\pgfqpoint{2.499456in}{2.822852in}}%
\pgfpathlineto{\pgfqpoint{2.521949in}{2.863332in}}%
\pgfpathlineto{\pgfqpoint{2.544441in}{2.948594in}}%
\pgfpathlineto{\pgfqpoint{2.566933in}{3.208217in}}%
\pgfpathlineto{\pgfqpoint{2.589425in}{3.289098in}}%
\pgfpathlineto{\pgfqpoint{2.611918in}{3.667933in}}%
\pgfpathlineto{\pgfqpoint{2.634410in}{3.699986in}}%
\pgfpathlineto{\pgfqpoint{2.656902in}{3.521094in}}%
\pgfpathlineto{\pgfqpoint{2.679394in}{2.694006in}}%
\pgfpathlineto{\pgfqpoint{2.701887in}{0.699705in}}%
\pgfpathlineto{\pgfqpoint{2.724379in}{2.756227in}}%
\pgfpathlineto{\pgfqpoint{2.746871in}{3.510264in}}%
\pgfpathlineto{\pgfqpoint{2.769363in}{3.530981in}}%
\pgfpathlineto{\pgfqpoint{2.791855in}{3.599227in}}%
\pgfpathlineto{\pgfqpoint{2.814348in}{3.483913in}}%
\pgfpathlineto{\pgfqpoint{2.836840in}{3.157102in}}%
\pgfpathlineto{\pgfqpoint{2.859332in}{3.129348in}}%
\pgfpathlineto{\pgfqpoint{2.881824in}{3.040661in}}%
\pgfpathlineto{\pgfqpoint{2.904317in}{2.627393in}}%
\pgfpathlineto{\pgfqpoint{2.926809in}{2.706147in}}%
\pgfusepath{stroke}%
\end{pgfscope}%
\begin{pgfscope}%
\pgfpathrectangle{\pgfqpoint{0.565124in}{0.549691in}}{\pgfqpoint{2.474146in}{3.300309in}}%
\pgfusepath{clip}%
\pgfsetrectcap%
\pgfsetroundjoin%
\pgfsetlinewidth{1.505625pt}%
\definecolor{currentstroke}{rgb}{1.000000,0.498039,0.054902}%
\pgfsetstrokecolor{currentstroke}%
\pgfsetdash{}{0pt}%
\pgfpathmoveto{\pgfqpoint{0.677585in}{0.906650in}}%
\pgfpathlineto{\pgfqpoint{0.700077in}{0.906397in}}%
\pgfpathlineto{\pgfqpoint{0.722569in}{0.921274in}}%
\pgfpathlineto{\pgfqpoint{0.745062in}{0.918756in}}%
\pgfpathlineto{\pgfqpoint{0.767554in}{0.927114in}}%
\pgfpathlineto{\pgfqpoint{0.790046in}{0.915675in}}%
\pgfpathlineto{\pgfqpoint{0.812538in}{0.929551in}}%
\pgfpathlineto{\pgfqpoint{0.835031in}{0.925861in}}%
\pgfpathlineto{\pgfqpoint{0.857523in}{0.944819in}}%
\pgfpathlineto{\pgfqpoint{0.880015in}{0.959524in}}%
\pgfpathlineto{\pgfqpoint{0.902507in}{0.946199in}}%
\pgfpathlineto{\pgfqpoint{0.925000in}{0.945843in}}%
\pgfpathlineto{\pgfqpoint{0.947492in}{0.943624in}}%
\pgfpathlineto{\pgfqpoint{0.969984in}{0.962628in}}%
\pgfpathlineto{\pgfqpoint{0.992476in}{0.939117in}}%
\pgfpathlineto{\pgfqpoint{1.014969in}{0.958386in}}%
\pgfpathlineto{\pgfqpoint{1.037461in}{0.955787in}}%
\pgfpathlineto{\pgfqpoint{1.059953in}{0.961559in}}%
\pgfpathlineto{\pgfqpoint{1.082445in}{0.969561in}}%
\pgfpathlineto{\pgfqpoint{1.104938in}{0.978873in}}%
\pgfpathlineto{\pgfqpoint{1.127430in}{0.991037in}}%
\pgfpathlineto{\pgfqpoint{1.149922in}{0.980529in}}%
\pgfpathlineto{\pgfqpoint{1.172414in}{0.966296in}}%
\pgfpathlineto{\pgfqpoint{1.194906in}{0.961605in}}%
\pgfpathlineto{\pgfqpoint{1.217399in}{0.944038in}}%
\pgfpathlineto{\pgfqpoint{1.239891in}{0.957523in}}%
\pgfpathlineto{\pgfqpoint{1.262383in}{0.978574in}}%
\pgfpathlineto{\pgfqpoint{1.284875in}{0.974941in}}%
\pgfpathlineto{\pgfqpoint{1.307368in}{0.989818in}}%
\pgfpathlineto{\pgfqpoint{1.329860in}{0.981632in}}%
\pgfpathlineto{\pgfqpoint{1.352352in}{1.001844in}}%
\pgfpathlineto{\pgfqpoint{1.374844in}{0.984599in}}%
\pgfpathlineto{\pgfqpoint{1.397337in}{1.007776in}}%
\pgfpathlineto{\pgfqpoint{1.419829in}{1.032610in}}%
\pgfpathlineto{\pgfqpoint{1.442321in}{1.045440in}}%
\pgfpathlineto{\pgfqpoint{1.464813in}{1.000694in}}%
\pgfpathlineto{\pgfqpoint{1.487306in}{0.987151in}}%
\pgfpathlineto{\pgfqpoint{1.509798in}{1.004373in}}%
\pgfpathlineto{\pgfqpoint{1.532290in}{1.021343in}}%
\pgfpathlineto{\pgfqpoint{1.554782in}{1.044475in}}%
\pgfpathlineto{\pgfqpoint{1.577275in}{1.037714in}}%
\pgfpathlineto{\pgfqpoint{1.599767in}{0.993210in}}%
\pgfpathlineto{\pgfqpoint{1.622259in}{1.066687in}}%
\pgfpathlineto{\pgfqpoint{1.644751in}{1.058501in}}%
\pgfpathlineto{\pgfqpoint{1.667243in}{1.046636in}}%
\pgfpathlineto{\pgfqpoint{1.689736in}{1.030816in}}%
\pgfpathlineto{\pgfqpoint{1.712228in}{1.044705in}}%
\pgfpathlineto{\pgfqpoint{1.734720in}{1.077896in}}%
\pgfpathlineto{\pgfqpoint{1.757212in}{1.091313in}}%
\pgfpathlineto{\pgfqpoint{1.779705in}{1.058892in}}%
\pgfpathlineto{\pgfqpoint{1.802197in}{1.045992in}}%
\pgfpathlineto{\pgfqpoint{1.824689in}{1.045406in}}%
\pgfpathlineto{\pgfqpoint{1.847181in}{1.102948in}}%
\pgfpathlineto{\pgfqpoint{1.869674in}{1.061708in}}%
\pgfpathlineto{\pgfqpoint{1.892166in}{1.140083in}}%
\pgfpathlineto{\pgfqpoint{1.914658in}{1.122619in}}%
\pgfpathlineto{\pgfqpoint{1.937150in}{1.058156in}}%
\pgfpathlineto{\pgfqpoint{1.959643in}{1.088588in}}%
\pgfpathlineto{\pgfqpoint{1.982135in}{1.088945in}}%
\pgfpathlineto{\pgfqpoint{2.004627in}{1.113491in}}%
\pgfpathlineto{\pgfqpoint{2.027119in}{1.145670in}}%
\pgfpathlineto{\pgfqpoint{2.049612in}{1.111134in}}%
\pgfpathlineto{\pgfqpoint{2.072104in}{1.103707in}}%
\pgfpathlineto{\pgfqpoint{2.094596in}{1.119699in}}%
\pgfpathlineto{\pgfqpoint{2.117088in}{1.119090in}}%
\pgfpathlineto{\pgfqpoint{2.139581in}{1.189474in}}%
\pgfpathlineto{\pgfqpoint{2.162073in}{1.175677in}}%
\pgfpathlineto{\pgfqpoint{2.184565in}{1.155052in}}%
\pgfpathlineto{\pgfqpoint{2.207057in}{1.171676in}}%
\pgfpathlineto{\pgfqpoint{2.229549in}{1.175091in}}%
\pgfpathlineto{\pgfqpoint{2.252042in}{1.190853in}}%
\pgfpathlineto{\pgfqpoint{2.274534in}{1.143360in}}%
\pgfpathlineto{\pgfqpoint{2.297026in}{1.240658in}}%
\pgfpathlineto{\pgfqpoint{2.319518in}{1.312629in}}%
\pgfpathlineto{\pgfqpoint{2.342011in}{1.352914in}}%
\pgfpathlineto{\pgfqpoint{2.364503in}{1.302431in}}%
\pgfpathlineto{\pgfqpoint{2.386995in}{1.278667in}}%
\pgfpathlineto{\pgfqpoint{2.409487in}{1.291118in}}%
\pgfpathlineto{\pgfqpoint{2.431980in}{1.461813in}}%
\pgfpathlineto{\pgfqpoint{2.454472in}{1.536784in}}%
\pgfpathlineto{\pgfqpoint{2.476964in}{1.474712in}}%
\pgfpathlineto{\pgfqpoint{2.499456in}{1.572183in}}%
\pgfpathlineto{\pgfqpoint{2.521949in}{1.420424in}}%
\pgfpathlineto{\pgfqpoint{2.544441in}{1.562112in}}%
\pgfpathlineto{\pgfqpoint{2.566933in}{1.545763in}}%
\pgfpathlineto{\pgfqpoint{2.589425in}{1.581633in}}%
\pgfpathlineto{\pgfqpoint{2.611918in}{1.612284in}}%
\pgfpathlineto{\pgfqpoint{2.634410in}{1.989452in}}%
\pgfpathlineto{\pgfqpoint{2.656902in}{1.517044in}}%
\pgfpathlineto{\pgfqpoint{2.679394in}{2.776439in}}%
\pgfpathlineto{\pgfqpoint{2.701887in}{0.699705in}}%
\pgfpathlineto{\pgfqpoint{2.724379in}{1.732324in}}%
\pgfpathlineto{\pgfqpoint{2.746871in}{1.353259in}}%
\pgfpathlineto{\pgfqpoint{2.769363in}{1.499304in}}%
\pgfpathlineto{\pgfqpoint{2.791855in}{1.171860in}}%
\pgfpathlineto{\pgfqpoint{2.814348in}{1.218331in}}%
\pgfpathlineto{\pgfqpoint{2.836840in}{1.244521in}}%
\pgfpathlineto{\pgfqpoint{2.859332in}{1.045843in}}%
\pgfpathlineto{\pgfqpoint{2.881824in}{0.941692in}}%
\pgfpathlineto{\pgfqpoint{2.904317in}{1.104695in}}%
\pgfpathlineto{\pgfqpoint{2.926809in}{0.766847in}}%
\pgfusepath{stroke}%
\end{pgfscope}%
\begin{pgfscope}%
\pgfsetrectcap%
\pgfsetmiterjoin%
\pgfsetlinewidth{0.803000pt}%
\definecolor{currentstroke}{rgb}{0.000000,0.000000,0.000000}%
\pgfsetstrokecolor{currentstroke}%
\pgfsetdash{}{0pt}%
\pgfpathmoveto{\pgfqpoint{0.565124in}{0.549691in}}%
\pgfpathlineto{\pgfqpoint{0.565124in}{3.850000in}}%
\pgfusepath{stroke}%
\end{pgfscope}%
\begin{pgfscope}%
\pgfsetrectcap%
\pgfsetmiterjoin%
\pgfsetlinewidth{0.803000pt}%
\definecolor{currentstroke}{rgb}{0.000000,0.000000,0.000000}%
\pgfsetstrokecolor{currentstroke}%
\pgfsetdash{}{0pt}%
\pgfpathmoveto{\pgfqpoint{3.039270in}{0.549691in}}%
\pgfpathlineto{\pgfqpoint{3.039270in}{3.850000in}}%
\pgfusepath{stroke}%
\end{pgfscope}%
\begin{pgfscope}%
\pgfsetrectcap%
\pgfsetmiterjoin%
\pgfsetlinewidth{0.803000pt}%
\definecolor{currentstroke}{rgb}{0.000000,0.000000,0.000000}%
\pgfsetstrokecolor{currentstroke}%
\pgfsetdash{}{0pt}%
\pgfpathmoveto{\pgfqpoint{0.565124in}{0.549691in}}%
\pgfpathlineto{\pgfqpoint{3.039270in}{0.549691in}}%
\pgfusepath{stroke}%
\end{pgfscope}%
\begin{pgfscope}%
\pgfsetrectcap%
\pgfsetmiterjoin%
\pgfsetlinewidth{0.803000pt}%
\definecolor{currentstroke}{rgb}{0.000000,0.000000,0.000000}%
\pgfsetstrokecolor{currentstroke}%
\pgfsetdash{}{0pt}%
\pgfpathmoveto{\pgfqpoint{0.565124in}{3.850000in}}%
\pgfpathlineto{\pgfqpoint{3.039270in}{3.850000in}}%
\pgfusepath{stroke}%
\end{pgfscope}%
\end{pgfpicture}%
\makeatother%
\endgroup%

		\end{center}
		\caption{$\mu_1 = 0.9$}
	\end{subfigure}
	\caption{Expected regret after $n=10000$ steps for Bernoulli bandits with $k=2$ arms and means $\mu_1 = 0.5$ in (a), $0.1$ in (b) and $0.9$ in (c), and $\mu_2 = 0.5 + \Delta$ with $\Delta \in [-0.5,0.5]$. The plots were averaged over 50 runs for each $\Delta$.}
	\label{fig:regret}
\end{figure}

\section{Regret Minimization in RL}

We consider a finite-horizon MDP $M^\star = (S, A, p_h, r_h)$ with stage-dependent transitions and rewards.

\begin{itemize}
	\item We define the event $\mathcal{E} = \{\forall k, M^\star \in \mathcal{M}_k\}$ and $\mathcal{M}_k = \{ M = (S,A, p_{h,k}, r_{h,k}) ~:~ r_{h,k}(s,a) \in \mathcal{B}^r_{h,k}(s,a), p_{h,k}(\cdot|s,a) \in \mathcal{B}^p_{h,k}(s,a)  \}$. Let us define confidence intervals $\beta_{hk}^r(s,a)$ and $\beta_{hk}^p(s,a)$ as a function of $\delta$ such that $\mathbb{P}(\neg\mathcal{E}) \leq \delta/2$. 
	
	Let us choose:
	\[
		\boxed{\beta_{hk}^r(s,a) = \sqrt{\frac{\log \left(\frac{8HSAK}{\delta}\right)}{2N_{h,k}(s,a)}}}
		\quad \text{and} \quad
		\boxed{\beta_{hk}^p(s,a) = \sqrt{\frac{2\log \left(\frac{4HSAK\left(2^S-2\right)}{\delta}\right)}{N_{h,k}(s,a)}}}		
	\]
	\begin{align*}
		\mathbb{P}(\neg\mathcal{E}) &= \mathbb{P}\left(\bigcup_{k=1}^K\left\{M^\star \notin \mathcal{M}_k\right\}\right) \\
		&= \mathbb{P}\left(\bigcup_{k=1}^K \bigcup_{h=1}^H \bigcup_{s=1}^S \bigcup_{a=1}^A \left\{|\wh{r}_{hk}(s,a) - r_h(s,a)| \geq \beta_{hk}^r(s,a) \right\} \cup \left\{\|\wh{p}_{hk}(\cdot|s,a) - p_{h}(\cdot|s,a)\|_1\geq \beta_{hk}^p(s,a) \right\}\right) \\
		&\leq \sum_{k=1}^K \sum_{h=1}^H \sum_{s=1}^S \sum_{a=1}^A \mathbb{P}\left\{|\wh{r}_{hk}(s,a) - r_h(s,a)| \geq \beta_{hk}^r(s,a) \right\} + \mathbb{P}\left\{\|\wh{p}_{hk}(\cdot|s,a) - p_{h}(\cdot|s,a)\|_1\geq \beta_{hk}^p(s,a) \right\} \\
		&\leq \sum_{k=1}^K \sum_{h=1}^H \sum_{s=1}^S \sum_{a=1}^A \left[ 2\exp\left(-2N_{h,k}(s,a)\beta_{hk}^r(s,a)^2\right) + (2^S - 2) \exp\left(- \frac{N_{h,k}(s,a) \beta_{hk}^p(s,a)^2}{2} \right) \right] \\
		&= \sum_{k=1}^K \sum_{h=1}^H \sum_{s=1}^S \sum_{a=1}^A \left[ \frac{\delta}{4HSAK} + \frac{\delta}{4HSAK} \right] \\
		&= \frac{\delta}{2}
	\end{align*}
	Therefore, $\boxed{\mathbb{P}(\neg\mathcal{E}) \leq \frac{\delta}{2}}$.
	
	
	\item Let us be under the event $\mathcal{E}$ and let $b_{h,k}(s,a)$ be a bonus to define.
	\[
		Q_{h,k}(s,a) = \wh{r}_{h,k}(s,a) + b_{h,k}(s,a) + \sum_{s'} \wh{p}_{h,k}(s'|s,a) V_{h+1,k}(s')
	\]
	Let us prove by induction that $\forall h,s,a,k,\; Q_{h,k}(s,a) \geq Q^\star_h(s,a)$
	
	\subitem \underline{Induction step}
	
	Let $h \in [1,H-1]$ and let us assume the following: $\quad \forall s,a,k,\; Q_{h+1,k}(s,a)\geq Q^\star_{h+1}(s,a)\quad$ (inductive assumption).

	Let us show that: $\quad \forall s,a,k,\; Q_{h,k}(s,a) \geq Q^\star_{h}(s,a)$.
	\begin{align*}
	Q_{h,k}(s,a) - Q^\star_h(s,a)
	&= \wh r_{h,k}(s,a)+b_{h,k}(s,a) + \sum_{s'} \wh{p}_{h,k}(s'|s,a)V_{h+1,k}(s') - \left(r_h(s,a)+\sum_{s'} p_h(s'|s,a)V^\star_{h+1}(s') \right) \\
	&= \sum_{s'}\Big( \wh{p}_{h,k}(s'|s,a)\min{\{H,\max_{a'} Q_{h+1,k}(s',a')\}} - p_h(s'|s,a)\max_{a'} Q^\star_{h+1}(s',a') \Big) \\ 
	&\qquad + \wh r_{h,k}(s,a) + b_{h,k}(s,a) - r_h(s,a) \\
	&\geq \sum_{s'}\Big( \wh{p}_{h,k}(s'|s,a)\min{\{H,\max_{a'} Q_{h+1,k}(s',a')\}} - p_h(s'|s,a)\min{\{H,\max_{a'} Q_{h+1,k}(s',a')\}} \Big) \\ 
	&\qquad + \wh r_{h,k}(s,a) + b_{h,k}(s,a) - r_h(s,a) \\
	&= \sum_{s'} \min{\{H,\max_{a'} Q_{h+1,k}(s',a')\}} \left( \wh{p}_{h,k}(s'|s,a) -  p_h(s'|s,a)\right) \\
	&\qquad + \wh r_{h,k}(s,a)+b_{h,k}(s,a) - r_h(s,a) \\
	&\geq - \sum_{s'} \min{\{H,\max_{a'} Q_{h+1,k}(s',a')\}} \left| \wh{p}_{h,k}(s'|s,a) -  p_h(s'|s,a)\right| \\
	&\qquad + \wh r_{h,k}(s,a)+b_{h,k}(s,a) - r_h(s,a) \\
	&\geq - H\sum_{s'} \left| \wh{p}_{h,k}(s'|s,a) -  p_h(s'|s,a)\right| + \wh r_{h,k}(s,a)+b_{h,k}(s,a) - r_h(s,a) \\
	&= - H \left \| \wh{p}_{h,k}(s'|s,a) -  p_h(s'|s,a) \right \|_1 + \wh r_{h,k}(s,a)+b_{h,k}(s,a) - r_h(s,a) \\
	&\geq - H\beta^p_{h,k}(s,a) + b_{h,k}(s,a) - \beta^r_{h,k}(s,a) + \underbrace{\wh r_{h,k}(s,a) + \beta^r_{h,k}(s,a) - r_h(s,a)}_{\geq 0} \\
	&\geq b_{h,k}(s,a) - \beta^r_{h,k}(s,a) - H\beta^p_{h,k}(s,a) \\
	\end{align*}
	
	Indeed, the induction step works if $b_{h,k}(s,a)$ is chosen such that
	\begin{equation*}
		b_{h,k}(s,a) \geq  \beta^r_{h,k}(s,a) + H\beta^p_{h,k}(s,a)
	\end{equation*}
	Let us define $b_{h,k}(s,a)$ to ensure $Q_{h,k}$ is optimistic.
	\begin{equation*}
		\boxed{b_{h,k}(s,a) = \beta^r_{h,k}(s,a) + H \beta^p_{h,k}(s,a)}
	\end{equation*}
	With this choice of $b_{h,k}(s,a)$,
	\[
		Q_{h,k}(s,a) - Q^\star_h(s,a) \geq 0
	\]
	The induction step is now proved.
	
	
	\subitem \underline{Base case}
	
	Since we are under the event $\mathcal{E}$, we have:
	\begin{equation*}
		\wh r_{H,k}(s,a)+ b_{Hk}(s,a) \geq \wh r_{H,k}(s,a)+\beta^r_{Hk}(s,a) \geq r_{H}(s,a).
	\end{equation*}
	Then, $\forall s',\; V_{H+1, k}(s') = V^\star_{H+1}(s') = 0$. Therefore, $\forall s,a,k, \; Q_{H,k}(s,a) - Q^\star_H(s,a)$. The base case is proven.
	
	Combining the base case and the inductive step gives us:
	\[
	\boxed{\forall h,s,a,k,\; Q_{h,k}(s,a) \geq Q^\star_h(s,a)}
	\]
	
	
	\item The aim in this question is to prove the following:
	\begin{equation}
	\label{eq:1}
	\delta_{1,k}(s_{1,k}) \leq \sum_{h=1}^H Q_{h,k}(s_{h,k},a_{h,k}) - r(s_{h,k},a_{h,k}) - \mathbb{E}_{Y\sim p(\cdot|s_{h,k},a_{h,k})}[V_{h+1,k}(Y)]) + m_{h,k}
	\end{equation}
	Where $\delta_{h,k}(s) = V_{h,k}(s) - V_h^{\pi_k}(s)$ and $m_{h,k} = \mathbb{E}_{Y\sim p(\cdot|s_{h,k},a_{h,k})}[\delta_{h+1,k}(Y)] - \delta_{h+1,k}(s_{h+1,k})$.
	
	\subitem 1. Let us show that $V^{\pi_{k}}_h(s_{h,k}) = r(s_{h,k},a_{h,k}) + \mathbb{E}_{p}[V_{h+1,k}(s')] - \delta_{h+1,k}(s_{h+1,k}) - m_{h,k}$.
	\begin{align*}
		\quad r&(s_{h,k},a_{h,k}) + \mathbb{E}_{p}[V_{h+1,k}(s')] - \delta_{h+1,k}(s_{h+1,k}) - m_{h,k} \\
		&= r(s_{h,k},a_{h,k}) + \mathbb{E}_{p}[V_{h+1,k}(s')] - \delta_{h+1,k}(s_{h+1,k}) - \left( \mathbb{E}_{Y\sim p(\cdot|s_{h,k},a_{h,k})}[\delta_{h+1,k}(Y)] - \delta_{h+1,k}(s_{h+1,k}) \right) \\
		&= r(s_{h,k},a_{h,k}) + \mathbb{E}_{p}[V_{h+1,k}(s')] - \mathbb{E}_{Y\sim p(\cdot|s_{h,k},a_{h,k})}[\delta_{h+1,k}(Y)] \\
		&= r(s_{h,k},a_{h,k}) + \mathbb{E}_{p}[V_{h+1,k}(s')] - \mathbb{E}_{Y\sim p(\cdot|s_{h,k},a_{h,k})}[V_{h+1,k}(Y) - V_{h+1}^{\pi_k}(Y)] \\
		&= r(s_{h,k},a_{h,k}) + \mathbb{E}_{p}[V_{h+1,k}(s')] + \mathbb{E}_{p}[V_{h+1}^{\pi_k}(Y)] - \mathbb{E}_{Y\sim p(\cdot|s_{h,k},a_{h,k})}[V_{h+1,k}(Y)] \\
		&= r(s_{h,k},a_{h,k}) + \mathbb{E}_{p}[V_{h+1}^{\pi_k}(Y)] \\
		&= V^{\pi_{k}}_h(s_{h,k}) \qquad \text{(Bellman equation)}
	\end{align*}
	\qquad \qquad Therefore, $\boxed{V^{\pi_{k}}_h(s_{h,k}) = r(s_{h,k},a_{h,k}) + \mathbb{E}_{p}[V_{h+1,k}(s')] - \delta_{h+1,k}(s_{h+1,k}) - m_{h,k}}$.
		
	\subitem 2. Let us prove that $V_{h,k}(s_{h,k}) \leq Q_{h,k}(s_{h,k},a_{h,k})$.
	\begin{align*}
		V_{h,k}(s_{h,k}) &= \min \{H, \max_{a'}Q_{h,k}(s_{h,k},a')\} \\
		& \leq \max_{a'}Q_{h,k}(s_{h,k},a') \\
		& \leq Q_{h,k}(s_{h,k},a_{h,k})		
	\end{align*}
	\qquad \qquad Therefore, $\boxed{V_{h,k}(s_{h,k}) \leq Q_{h,k}(s_{h,k},a_{h,k})}$.
	
	\subitem 3. Let us prove equation~\ref{eq:1}.
	\begin{align*}
		\delta_{1,k}(s_{1,k}) &= V_{1,k}(s_{1,k}) - V_1^{\pi_k}(s_{1,k}) \\
		&\leq Q_{1,k}(s_{1,k},a_{1,k}) - \left( r(s_{1,k},a_{1,k}) + \mathbb{E}_{p}[V_{2,k}(s')] - \delta_{2,k}(s_{2,k}) - m_{1,k}\right) \\
		&= \delta_{2,k}(s_{2,k}) + \big[Q_{1,k}(s_{1,k},a_{1,k}) - r(s_{1,k},a_{1,k}) - \mathbb{E}_{p}[V_{2,k}(s')] - m_{1,k}\big] \\
		&= V_{2,k}(s_{2,k}) - V_2^{\pi_k}(s_{2,k}) + \big[Q_{1,k}(s_{1,k},a_{1,k}) - r(s_{1,k},a_{1,k}) - \mathbb{E}_{p}[V_{2,k}(s')] - m_{1,k}\big] \\
		&\leq \dots \\
		&\leq \sum_{h=1}^{H} Q_{h,k}(s_{h,k},a_{h,k}) - r(s_{h,k},a_{h,k}) - \mathbb{E}_{p}[V_{h+1,k}(s')] - m_{h,k}
	\end{align*}
	\qquad \qquad Therefore, $\boxed{\delta_{1,k}(s_{1,k}) \leq \sum_{h=1}^H Q_{h,k}(s_{h,k},a_{h,k}) - r(s_{h,k},a_{h,k}) - \mathbb{E}_{Y\sim p(\cdot|s_{h,k},a_{h,k})}[V_{h+1,k}(Y)]) + m_{h,k}}$.
	
	
	\item Let us show that with probability $1-\delta$, $R(T) \leq \sum_{k,h} b_{h,k}(s_{h,k},a_{h,k}) + 2H\sqrt{KH \log(2/\delta)}$
	\begin{align*}
		R(T) &= \sum_{k=1}^{K} V_1^\star(s_{1,k}) - V_1^{\pi_k}(s_{1,k}) \\
		&= \sum_{k=1}^{K} V_1^{\pi_k^\star}(s_{1,k}) - V_{1,k}(s_{1,k}) + \sum_{k=1}^{K} V_{1,k}(s_{1,k}) - V_1^{\pi_k}(s_{1,k}) \\
		&= \sum_{k=1}^{K} -\delta_{1,k}^\star(s_{1,k}) + \delta_{1,k}(s_{1,k}) \\
		&\leq \sum_{k=1}^{K} \delta_{1,k}(s_{1,k}) \quad \left(\text{Since}\; V \geq V^\star \geq V^{\pi_k} \right) \\
		&\leq \sum_{k=1}^{K} \sum_{h=1}^H Q_{h,k}(s_{h,k},a_{h,k}) - r(s_{h,k},a_{h,k}) - \mathbb{E}_{Y\sim p(\cdot|s_{h,k},a_{h,k})}[V_{h+1,k}(Y)]) + m_{h,k} \\
		&= \sum_{k=1}^{K} \sum_{h=1}^H m_{h,k} + \sum_{k=1}^{K} \sum_{h=1}^H Q_{h,k}(s_{h,k},a_{h,k}) - r(s_{h,k},a_{h,k}) - \mathbb{E}_{Y\sim p(\cdot|s_{h,k},a_{h,k})}[V_{h+1,k}(Y)]) \\
		&\leq \sum_{k=1}^{K} \sum_{h=1}^H m_{h,k} + \sum_{k=1}^{K} \sum_{h=1}^H Q_{h,k}(s_{h,k},a_{h,k}) - r(s_{h,k},a_{h,k}) - \mathbb{E}_{Y\sim p(\cdot|s_{h,k},a_{h,k})}[V^\star_{h+1}(Y)]) \\
		&= \sum_{k=1}^{K} \sum_{h=1}^H m_{h,k} + \sum_{k=1}^{K} \sum_{h=1}^H Q_{h,k}(s_{h,k},a_{h,k}) - Q^\star_{h,k}(s_{h,k},a_{h,k}) \\
	\end{align*}
	The first sum is bounded by Azuma with probability $1 - \frac{\delta}{2}$ whereas the second one is bounded by the bonuses again with probability $1 - \frac{\delta}{2}$. Therefore, with probability $1 - \delta$, we have:
	\begin{equation*}
		\boxed{R(T) \leq \sum_{k=1}^{K} \sum_{h=1}^H b_{h,k}(s_{h,k},a_{h,k}) + 2H\sqrt{KH \log(2/\delta)}}
	\end{equation*}
	
	
	\item Finally, let us show that $R(T) \lesssim H^2S\sqrt{AK}$.
	\begin{align*}
		\sum_{h,k} \frac{1}{\sqrt{N_{h,k}(s_{h,k},a_{h,k})}} &= \sum_{h=1}^H\sum_{s,a} \sum_{i=1}^{N_{h,K}(s,a)} \frac{1}{\sqrt{i}} \\
		&\leq 2\sum_{h=1}^H\sum_{s,a} \sqrt{N_{h,K}(s,a)} \\
		&\leq 2\sqrt{SAH}\sqrt{\sum_{h=1}^H\sum_{s,a} N_{h,K}(s,a)} \qquad \left(\text{Jensen}\right)\\
		&\leq 2\sqrt{SAH}\sqrt{\sum_{h=1}^H K} \qquad \left(\forall h, \; \sum_{s,a} N_{h,K}(s,a) \leq K \right) \\
		&\leq 2H\sqrt{SAK}
	\end{align*}
	Therefore, 
	\[
	\boxed{\sum_{k=1}^{K} \sum_{h=1}^H \frac{1}{\sqrt{N_{h,k}(s_{h,k},a_{h,k})}} \leq 2H\sqrt{SAK}}
	\]
	Let us conclude on the regret.
	\begin{align*}
		R(T) &\leq \sum_{k=1}^{K} \sum_{h=1}^H b_{h,k}(s_{h,k},a_{h,k}) + 2H\sqrt{KH \log(2/\delta)} \\
		&= \left( \sqrt{\frac{\log \left(\frac{8HSAK}{\delta}\right)}{2}} + H\sqrt{2\log \left(\frac{4HSAK\left(2^S-2\right)}{\delta}\right)} \right) \sum_{k=1}^{K} \sum_{h=1}^H \frac{1}{\sqrt{N_{h,k}(s_{h,k},a_{h,k})}} + 2H\sqrt{KH \log(2/\delta)} \\
		&\leq \left( \sqrt{\frac{\log \left(\frac{8HSAK}{\delta}\right)}{2}} + H\sqrt{2\log \left(\frac{4HSAK\left(2^S-2\right)}{\delta}\right)} \right) 2H\sqrt{SAK}  + 2H\sqrt{KH \log(2/\delta)}\\
		&\leq f\left(H,S,A,K\right) H\sqrt{SAK} + cH\sqrt{KH} \\
	\end{align*}
	Where $c$ is a constant (depends on $\delta$) and $f\left(H,S,A,K\right) \lesssim H\sqrt{S}$.
	Therefore, we find the regret upper bound:
	\[
	\boxed{R(T) \lesssim H^2S\sqrt{AK}}
	\]
\end{itemize}

\end{document}
